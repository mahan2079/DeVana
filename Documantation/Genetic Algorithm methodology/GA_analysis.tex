\documentclass[12pt,a4paper]{article}
\usepackage[utf8]{inputenc}
\usepackage[T1]{fontenc}
\usepackage{amsmath,amsfonts,amssymb}
\usepackage{graphicx}
\usepackage{float}
\usepackage{geometry}
\usepackage{tikz}
\usepackage{algorithm}
\usepackage{algorithmic}
\usepackage{listings}
\usepackage{xcolor}
\usepackage{hyperref}
\usepackage{booktabs}
\usepackage{multirow}
\usepackage{array}
\usepackage{subcaption}
\usepackage{enumitem}
\usepackage{setspace}
\usepackage{stackengine}

% Glossaries and acronyms
\usepackage[acronym]{glossaries}
\makeglossaries

% Project metadata and convenience macros
\newcommand{\softwareName}{DeVana}
\newcommand{\softwareVersion}{v0.4.1}

% Hyperref setup
\hypersetup{
    pdftitle={Advanced Genetic Algorithm Optimization for Dynamic Vibration Absorber Design: A Comprehensive Methodology},
    pdfauthor={Master's Thesis Methodology},
    pdfsubject={\softwareName{} \softwareVersion{} open-source thesis},
    colorlinks=true,
    linkcolor=blue,
    citecolor=blue,
    urlcolor=blue
}

% Acronyms
\newacronym{dva}{DVA}{Dynamic Vibration Absorber}
\newacronym{DVA}{DVA}{Dynamic Vibration Absorber} % for existing \gls{DVA} usage
\newacronym{frf}{FRF}{Frequency Response Function}
\newacronym{ga}{GA}{Genetic Algorithm}
\newacronym{pso}{PSO}{Particle Swarm Optimization}
\newacronym{sa}{SA}{Simulated Annealing}
\newacronym{de}{DE}{Differential Evolution}
\newacronym{cmaes}{CMA-ES}{Covariance Matrix Adaptation Evolution Strategy}
\newacronym{rl}{RL}{Reinforcement Learning}
\newacronym{knn}{KNN}{k-Nearest Neighbors}
\newacronym{hdi}{HDI}{Highest Density Interval}
\newacronym{ucb}{UCB}{Upper Confidence Bound}
\newacronym{ei}{EI}{Expected Improvement}
\newacronym{gui}{GUI}{Graphical User Interface}
\newacronym{kpi}{KPI}{Key Performance Indicator}
\newacronym{qmc}{QMC}{Quasi-Monte Carlo}

% Page setup
\geometry{margin=2.5cm}
\onehalfspacing

% Code listing setup
\lstset{
    basicstyle=\ttfamily\footnotesize,
    breaklines=true,
    frame=single,
    numbers=left,
    numberstyle=\tiny,
    keywordstyle=\color{blue},
    commentstyle=\color{green!60!black},
    stringstyle=\color{red},
    backgroundcolor=\color{gray!10}
}

% Title and author
\title{Advanced Genetic Algorithm Optimization for Dynamic Vibration Absorber Design: A Comprehensive Methodology}
\author{Master's Thesis Methodology}
\date{\today}

\begin{document}

\maketitle

\begin{abstract}
\softwareName{} (\softwareVersion{}) is an open-source, first-of-its-kind playground for the design of \glspl{dva} that unifies classical and modern optimization under one reproducible, extensible framework. The platform integrates a rigorous mechanical model (fully coupled 2DOF--3DOF), an \gls{frf}-based fitness pipeline, and a suite of optimizers (\gls{ga}, \gls{pso}, \gls{sa}, \gls{cmaes}, \gls{rl}, and \gls{de}). Beyond providing many algorithms, \softwareName{} contributes three methodological advances tailored to vibration absorber design: (i) adaptive operator control for \gls{ga} and \gls{pso} via ML-bandit and \gls{rl} controllers that tune parameters online; (ii) surrogate-assisted screening using a \gls{knn} predictor to reduce expensive \gls{frf} evaluations; and (iii) advanced population seeding (random, Sobol, Latin hypercube, and a NeuralSeeder with \gls{ucb}/\gls{ei} acquisition) for improved exploration. The system instruments each run with detailed metrics and provides statistical post-analysis across repeated runs to recommend robust parameter ranges for every DVA variable. We document the mathematical formulation, software architecture, algorithmic controllers, benchmarking protocol, and statistical synthesis pipeline, and we release the full implementation to accelerate research and practice in vibration control.
\end{abstract}

\noindent\textbf{Keywords:} Dynamic vibration absorber, frequency response function, genetic algorithm, CMA-ES, PSO, simulated annealing, differential evolution, reinforcement learning, surrogate modeling, seeding, robustness, reproducibility, open-source.

\vspace{0.5em}
\noindent\textbf{Software:} \softwareName{} \softwareVersion{} (open-source; available on GitHub and project website).

\tableofcontents
\newpage

% List of acronyms
\printglossary[type=\acronymtype,title=List of Acronyms]
\newpage

\section{Introduction and Motivation}

This thesis presents \softwareName{}, a comprehensive optimization framework and software platform for designing \glspl{dva} with practical applicability in mechanical engineering. The framework addresses the challenges of vast configuration spaces, multi-criteria optimization, and computational efficiency by (i) supporting multiple optimizers within a single configurable playground, (ii) introducing adaptive controllers for \gls{ga} that learn online, (iii) reducing evaluation cost via surrogate-assisted screening, and (iv) enabling statistical synthesis of recommended parameter ranges across repeated runs.

\subsection{Contributions and novelties}
\label{subsec:contributions}
\begin{enumerate}
    \item \textbf{Unified optimization playground:} A modular, open-source platform that allows designers to configure and compare \gls{ga}, \gls{pso}, \gls{sa}, \gls{cmaes}, \gls{rl}, and \gls{de} on identical problems with consistent instrumentation.
    \item \textbf{Adaptive \gls{ga} controllers:} Online tuning of crossover, mutation, and population size via an ML-bandit controller and a compact \gls{rl} agent, overcoming fixed-operator limitations.
    \item \textbf{Surrogate-assisted screening:} A \gls{knn}-based pre-filter that reduces the number of costly \gls{frf} evaluations while preserving exploration through novelty.
    \item \textbf{Advanced seeding:} Random, Sobol, and Latin hypercube \gls{qmc} options plus a NeuralSeeder that proposes individuals using \gls{ucb}/\gls{ei}, improving early search quality.
    \item \textbf{Statistical range synthesis:} A robust pipeline that aggregates multiple independent runs and outputs recommended parameter ranges per DVA variable using complementary criteria (IQR, P5--P95, Tukey, \gls{hdi}, Top-$q$, TMAD).
    \item \textbf{Reproducible engineering workflow:} Versioned release (\softwareVersion{}), open dataset of metrics, and exportable configurations for repeatable studies and fair comparisons.
\end{enumerate}

\subsection{Software availability and versioning}
\label{subsec:availability}
\softwareName{} is released under an open-source license at GitHub and the project website. This thesis documents \softwareVersion{}. The repository includes source code, example configurations, and instructions to reproduce all experiments presented herein.

\subsection{Thesis roadmap}
\label{subsec:roadmap}
We begin with the DVA configuration and parameter spaces, followed by the mechanical model and its \gls{frf}-based evaluation. We then detail the optimization problem, the traditional \gls{ga} baseline, the enhanced \gls{pso} methodology, and the advanced features. Subsequent sections describe software architecture, the algorithm suite and configuration playground, benchmarking and robustness analysis, statistical range synthesis, and the overall results, before concluding with reproducibility notes and future work.


\section{Software architecture and optimization playground}
\label{sec:architecture}

\subsection{Design principles}
\begin{itemize}
    \item \textbf{Modularity}: separable layers for mechanics, optimization, controllers, seeding, surrogate, \gls{gui}, and benchmarking.
    \item \textbf{Reproducibility}: versioned configurations, deterministic seeds on demand, and exportable results.
    \item \textbf{Extensibility}: clean interfaces to add new optimizers, surrogates, and metrics with minimal code changes.
    \item \textbf{Observability}: rich metrics, logs, and artifacts captured per run and per generation.
\end{itemize}

\subsection{System overview}
\begin{enumerate}
    \item \textbf{Mechanical core}: builds the mass, damping, stiffness, and force operators and computes \gls{frf}s per Sec.~\ref{subsubsec:chosen_performance_criteria_combined}.
    \item \textbf{Fitness pipeline}: evaluates Eq.~\eqref{Eq.objective_function_detailed} using modal, inter-modal, and global criteria with user-defined weights.
    \item \textbf{Algorithm suite}: \gls{ga}, \gls{pso}, \gls{sa}, \gls{cmaes}, \gls{rl}, and \gls{de}, with consistent bounds, fixed-parameter handling, and termination.
    \item \textbf{Controllers}: ML-bandit and \gls{rl} for adaptive \gls{ga} and \gls{pso} operator control (Sec.~\ref{Eq.ucb}--\ref{Eq.bandit_reward}).
    \item \textbf{Seeding}: Random, Sobol, LHS, and NeuralSeeder (\gls{ucb}/\gls{ei}) as in Sec.~\ref{Eq.harmonic.solution.2dof3dof} onward.
    \item \textbf{Surrogate screening}: \gls{knn}-based pre-filter with novelty (Sec.~\ref{sec:benchmarking}).
    \item \textbf{Instrumentation}: per-generation and system \gls{kpi}s (Sec.~\ref{sec:benchmarking}).
    \item \textbf{GUI}: interactive configuration, execution control, visualization (responses, convergence, ranges), and export.
\end{enumerate}

\subsection{Optimization playground}
Users configure experiments by selecting the optimizer, bounds and fixed masks, objective weights, seeding, controllers, surrogate settings, and stopping criteria. The same mechanical problem can thus be solved by multiple methods, enabling apples-to-apples comparisons with common metrics and visualizations.

\subsection{Data, logging, and metrics}
At each generation we record time, evaluations, fitness statistics, operator rates, population size, and resource usage, along with artifacts (best solutions, FRFs, convergence traces). Artifacts are exportable as CSV/JSON for downstream analysis.

\subsection{Extensibility}
New optimizers, surrogates, or controllers implement small interfaces (ask, tell/evaluate, and adapt hooks). New performance metrics can be registered and automatically included in the logging and reporting layers.

\subsection{Reproducibility}
All experiments store configuration hashes and software version (\softwareVersion{}), aiding peer reproduction. We provide example configurations used for this thesis alongside raw metric exports.


\section{Configuration Space of DVA Systems}

The design of \glspl{DVA} involves selecting appropriate combinations of mechanical components—masses, springs, dampers, and inerters—to attach to a primary system for vibration mitigation. Each component contributes to the system's dynamic behavior, and their combinations result in a multitude of possible configurations.

\subsection{Definition of Components and Parameters}

Let:

\begin{itemize}
    \item $\mathcal{M} = \{ m_i \mid m_i \in [m_i^{\min}, m_i^{\max}],\ i=1,\ldots,n_m \}$: Set of mass elements with viable ranges.
    \item $\mathcal{K} = \{ k_j \mid k_j \in [k_j^{\min}, k_j^{\max}],\ j=1,\ldots,n_k \}$: Set of spring elements with viable stiffness ranges.
    \item $\mathcal{C} = \{ c_l \mid c_l \in [c_l^{\min}, c_l^{\max}],\ l=1,\ldots,n_c \}$: Set of damping elements with viable damping coefficient ranges.
    \item $\mathcal{B} = \{ b_p \mid b_p \in [b_p^{\min}, b_p^{\max}],\ p=1,\ldots,n_b \}$: Set of inerter elements with viable inertance ranges.
\end{itemize}

Each parameter is bounded within a feasible design range, reflecting practical engineering constraints such as material properties, geometric limitations, and manufacturing capabilities.

\subsection{Total Number of Configurations}



\begin{align}
N_{\text{total}} = \left( n_m + n_k + n_c + n_b \right)\\
N_{\text{config}} = \frac{N_{\text{total}} \times (N_{\text{total}} + 1)}{2}
\label{eq:N_config}
\end{align}

This growth underscores the impracticality of exhaustively evaluating every possible configuration due to computational limitations.


\section{Parameter Space and Design Variables}

For a given configuration $s$, the parameter vector $\boldsymbol{\theta}_s$ comprises the design variables associated with the included components:

\begin{equation}
\boldsymbol{\theta}_s = [ \theta_1, \theta_2, \ldots, \theta_{n_s} ]^\top
\label{eq:theta_s}
\end{equation}

where $n_s$ is the number of parameters in configuration $s$, and each $\theta_i$ corresponds to a component parameter (e.g., mass, stiffness, damping coefficient, or inertance) within its viable range:

\begin{equation}
\theta_i \in [ \theta_i^{\min}, \theta_i^{\max} ]
\label{eq:theta_i_range}
\end{equation}

The feasible parameter space for configuration $s$ is thus defined as:

\begin{equation}
\Theta_s = \prod_{i=1}^{n_s} [ \theta_i^{\min}, \theta_i^{\max} ]
\label{eq:Theta_s}
\end{equation}




\section{Problem Statement}

\subsection{Mechanical System Definition}

\subsubsection{Fully Coupled 2DOF - 3DOF System Setup}
\paragraph{System Overview and Configuration}

In order to comprehend the fundamental concepts and operational principles of the advanced genetic algorithm methodology introduced in this work, a comprehensive, step-by-step analysis of the mechanical system is essential. The system under investigation is a sophisticated fully coupled 2DOF - 3DOF system, where "fully coupled" signifies that all system components are interconnected through masses, springs, dampers, and inerters, creating a complete vibrational system with comprehensive dynamic interactions.

The primary structure consists of a 2DOF main system with two primary masses representing different structural elements, augmented by three strategically positioned 1DOF Dynamic Vibration Absorbers (DVAs). This configuration allows for multi-modal vibration control targeting multiple resonant frequencies simultaneously. The system architecture includes:

\begin{itemize}
    \item \textbf{Primary structural elements}: Two primary masses ($M_1$, $M_2$) representing the main structural components with their inertial properties
    \item \textbf{Dynamic Vibration Absorbers}: Three DVA masses ($\mu_1$, $\mu_2$, $\mu_3$) with independent dynamic characteristics
    \item \textbf{Base excitations}: Lower and upper base motions providing external vibrational inputs
    \item \textbf{External forcing}: Direct force inputs applied to the primary masses
    \item \textbf{Complete coupling}: All components interconnected through mass, stiffness, damping, and inertial elements
\end{itemize}

The system's complexity arises from the extensive parameter space and coupling mechanisms. The complete system comprises 48 independent design parameters distributed across:
\begin{itemize}
    \item 15 mass coupling parameters ($\beta_1$ through $\beta_{15}$) for inertial interconnections
    \item 15 stiffness parameters ($\lambda_1$ through $\lambda_{15}$) for elastic coupling
    \item 15 damping parameters ($\nu_1$ through $\nu_{15}$) for energy dissipation
    \item 3 DVA mass parameters ($\mu_1$, $\mu_2$, $\mu_3$) for absorber sizing
\end{itemize}

This high-dimensional parameter space necessitates advanced optimization techniques capable of efficiently exploring the design domain while maintaining computational tractability.

\paragraph{Vibrational Modeling of the System and Assumptions}

The main 2DOF system is modeled using a comprehensive framework comprising masses, springs, dampers, and inerters, interconnected with both upper and lower bases. The modeling approach incorporates the following fundamental assumptions:

\begin{itemize}
    \item \textbf{Linear elastic behavior}: All stiffness elements ($K_1$, $K_2$, $K_3$) exhibit linear force-displacement relationships within the operational range
    \item \textbf{Linear viscous damping}: All damping elements ($C_1$, $C_2$, $C_3$) follow linear velocity-dependent force relationships
    \item \textbf{Point mass representation}: All masses are treated as point masses with concentrated inertial properties
    \item \textbf{One-dimensional motion}: All system components move in a single translational direction
    \item \textbf{Time-invariant parameters}: All system parameters remain constant during operation
    \item \textbf{No geometric nonlinearities}: The system operates within the linear regime, excluding geometric stiffening effects
    \item \textbf{Perfect bonding}: All interconnections between components are assumed to be rigid and perfect
\end{itemize}

The base excitations are modeled with flexibility to represent various mechanical scenarios:
\begin{itemize}
    \item \textbf{Foundation motion}: Representing actual physical ground motion or support excitation
    \item \textbf{Additional system modes}: Modeling extra degrees of freedom from more complex structures
    \item \textbf{Boundary condition variations}: Accommodating different support and mounting conditions
\end{itemize}

Each DVA is designed as an independent 1DOF system with its own mass, stiffness, damping, and inertial elements. The coupling between the primary system and DVAs, as well as between DVAs themselves, is achieved through comprehensive interconnection elements ensuring complete dynamic coupling.

\paragraph{Derivation of Governing Equations}

The governing equations for the fully coupled 2DOF-3DOF system are derived using Newton's method, applying Newton's second law to each degree of freedom while accounting for all interconnecting forces. The system's generalized coordinates are defined as:

\begin{align}\label{Eq.generalized.coordinate.2dof3dof}
    \mathbf{q} =
    \begin{bmatrix}
        U_1(t) \\
        U_2(t) \\
        u_1(t) \\
        u_2(t) \\
        u_3(t)
    \end{bmatrix}; \quad
    \dot{\mathbf{q}} =
    \begin{bmatrix}
        \dot{U_1}(t) \\
        \dot{U_2}(t) \\
        \dot{u_1}(t) \\
        \dot{u_2}(t) \\
        \dot{u_3}(t)
    \end{bmatrix}; \quad
    \ddot{\mathbf{q}} =
    \begin{bmatrix}
        \ddot{U_1}(t) \\
        \ddot{U_2}(t) \\
        \ddot{u_1}(t) \\
        \ddot{u_2}(t) \\
        \ddot{u_3}(t)
    \end{bmatrix}
\end{align}

where:
\begin{itemize}
    \item $U_1(t)$, $U_2(t)$: Displacements of the primary masses at time $t$
    \item $u_1(t)$, $u_2(t)$, $u_3(t)$: Displacements of the DVA masses at time $t$
\end{itemize}

The equations of motion are expressed in matrix form as:

\begin{equation} \label{Eq.EOM_dimensional_combined}
    \mathbf{M} \ddot{\mathbf{q}} + \mathbf{C} \dot{\mathbf{q}} + \mathbf{K} \mathbf{q} = \mathbf{F}(t)
\end{equation}

where:

\begin{itemize}
    \item $\mathbf{q}$: Generalized displacement vector, capturing displacements of the primary and \gls{DVA} masses.
    \item $\mathbf{M}$: Mass matrix.
    \item $\mathbf{C}$: Damping matrix.
    \item $\mathbf{K}$: Stiffness matrix.
    \item $\mathbf{F}(t)$: External force vector, which includes both external loads and base motion effects.
\end{itemize}

The generalized coordinate vector is defined as:

\begin{equation}\label{Eq.generalized_coordinate_combined}
    \mathbf{q} = 
  \begin{bmatrix}
    U_1 \\ U_2 \\ u_1 \\ u_2 \\ u_3
  \end{bmatrix}
\end{equation}

\paragraph{Mass Matrix}

\begin{equation}\label{Eq.mass_matrix_dimensional_combined}
\begin{aligned}
[M] =& 
\begin{bmatrix}
\shortstack[c]{$M_1 + b_1$ \\ $+ b_2 + b_3$} & 0 & \shortstack[c]{$-b_1$ \\ \,} & \shortstack[c]{$-b_2$ \\ \,} & \shortstack[c]{$-b_3$ \\ \,} \\
0 & \shortstack[c]{$M_2 + b_4$ \\ $+ b_5 + b_6$} & \shortstack[c]{$-b_4$ \\ \,} & \shortstack[c]{$-b_5$ \\ \,} & \shortstack[c]{$-b_6$ \\ \,} \\
\shortstack[c]{$-b_1$ \\ \,} & \shortstack[c]{$-b_4$ \\ \,} & \shortstack[c]{$m_1 + b_1$ \\ $+ b_4 + b_7$ \\ $+ b_8 + b_9$ \\ $+ b_{10}$} & \shortstack[c]{$-b_9$ \\ \,} & \shortstack[c]{$-b_{10}$ \\ \,} \\
\shortstack[c]{$-b_2$ \\ \,} & \shortstack[c]{$-b_5$ \\ \,} & \shortstack[c]{$-b_9$ \\ \,} & \shortstack[c]{$m_2 + b_2$ \\ $+ b_5 + b_9$ \\ $+ b_{11} + b_{12}$} & \shortstack[c]{$-b_{15}$ \\ \,} \\
\shortstack[c]{$-b_3$ \\ \,} & \shortstack[c]{$-b_6$ \\ \,} & \shortstack[c]{$-b_{10}$ \\ \,} & \shortstack[c]{$-b_{15}$ \\ \,} & \shortstack[c]{$m_3 + b_3$ \\ $+ b_6 + b_{10}$ \\ $+ b_{13} + b_{14}$ \\ $+ b_{15}$}
\end{bmatrix}
\end{aligned}
\end{equation}

\paragraph{Damping Matrix}

\begin{equation}\label{Eq.damping_matrix_dimensional_combined}
\begin{aligned}
[C] =& 
\begin{bmatrix}
\shortstack{$C_1 + C_2$ \\ $+ C_3$} & -C_3 & \shortstack{$-c_1$ \\ \,} & \shortstack{$-c_2$ \\ \,} & \shortstack{$-c_3$ \\ \,} \\
-C_3 & \shortstack{$C_3 + C_4$ \\ $+ C_5 + c_4$ \\ $+ c_5 + c_6$} & \shortstack{$-c_4$ \\ \,} & \shortstack{$-c_5$ \\ \,} & \shortstack{$-c_6$ \\ \,} \\
-c_1 & -c_2 & \shortstack{$c_1 + c_4$ \\ $+ c_7 + c_8$ \\ $+ c_9 + c_{10}$} & \shortstack{$-c_9$ \\ \,} & \shortstack{$-c_{10}$ \\ \,} \\
-c_2 & -c_5 & -c_9 & \shortstack{$c_2 + c_5$ \\ $+ c_9 + c_{11}$ \\ $+ c_{12} + c_{15}$} & \shortstack{$-c_{15}$ \\ \,} \\
-c_3 & -c_{6} & -c_{10} & -c_{15} & \shortstack{$c_3 + c_6$ \\ $+ c_{10} + c_{13}$ \\ $+ c_{14} + c_{15}$}
\end{bmatrix}
\end{aligned}
\end{equation}

\paragraph{Stiffness Matrix}

\begin{equation}\label{Eq.stiffness_matrix_dimensional_combined}
\begin{aligned}
[K] =& 
\begin{bmatrix}
\shortstack{$K_1 + K_2$ \\ $+ K_3$} & -K_3 & \shortstack{$-k_1$ \\ \,} & \shortstack{$-k_2$ \\ \,} & \shortstack{$-k_3$ \\ \,} \\
-K_3 & \shortstack{$K_3 + K_4$ \\ $+ K_5 + k_4$ \\ $+ k_5 + k_6$} & \shortstack{$-k_4$ \\ \,} & \shortstack{$-k_5$ \\ \,} & \shortstack{$-k_6$ \\ \,} \\
-k_1 & -k_2 & \shortstack{$k_1 + k_4$ \\ $+ k_7 + k_8$ \\ $+ k_9 + k_{10}$} & \shortstack{$-k_9$ \\ \,} & \shortstack{$-k_{10}$ \\ \,} \\
-k_2 & -k_5 & -k_9 & \shortstack{$k_2 + k_5$ \\ $+ k_9 + k_{11}$ \\ $+ k_{12} + k_{15}$} & \shortstack{$-k_{15}$ \\ \,} \\
-k_3 & -k_{6} & -k_{10} & -k_{15} & \shortstack{$k_3 + k_6$ \\ $+ k_{10} + k_{13}$ \\ $+ k_{14} + k_{15}$}
\end{bmatrix}
\end{aligned}
\end{equation}

\paragraph{Force Vector}

\begin{equation}\label{Eq.force_vector_dimensional_combined}
\begin{aligned}
[F] = & \begin{bmatrix}
F_1(t) + C_1 \dot{U}_{low} + C_2 \dot{U}_{upp} + K_1 U_{low} + K_2 U_{upp} \\
F_2(t) + C_4 \dot{U}_{low} + C_5 \dot{U}_{upp} + K_4 U_{low} + K_5 U_{upp} \\
\beta_7 \ddot{U}_{low} + \beta_8 \ddot{U}_{upp} + 2 \zeta_{dc} \omega_{dc} (\nu_7 \dot{U}_{low} + \nu_8 \dot{U}_{upp}) + \omega_{dc}^2 (\lambda_7 U_{low} + \lambda_8 U_{upp}) \\
\beta_{11} \ddot{U}_{low} + \beta_{12} \ddot{U}_{upp} + 2 \zeta_{dc} \omega_{dc} (\nu_{11} \dot{U}_{low} + \nu_{12} \dot{U}_{upp}) + \omega_{dc}^2 (\lambda_{11} U_{low} + \lambda_{12} U_{upp}) \\
\beta_{13} \ddot{U}_{low} + \beta_{14} \ddot{U}_{upp} + 2 \zeta_{dc} \omega_{dc} (\nu_{13} \dot{U}_{low} + \nu_{14} \dot{U}_{upp}) + \omega_{dc}^2 (\lambda_{13} U_{low} + \lambda_{14} U_{upp})
\end{bmatrix}
\end{aligned}
\end{equation}

\subsubsection{Dimensionless Form}

To simplify analysis, the system is normalized using dimensionless parameters listed in Table~\ref{Tab:dimensionless.table}.


\begin{table}[h!]
\centering
\caption{Dimensionless Parameters for System Normalization}
\label{Tab:dimensionless.table}
\begin{tabular}{lcc}
\toprule
\textbf{Parameter Group} & \textbf{Parameter} & \textbf{Definition} \\
\midrule
\textbf{Mass Ratios} & \( \Gamma \) & \( \Gamma = \frac{M_2}{M_1} \) \\
 & \( \mu_i \) & \( \mu_i = \frac{m_i}{M_1} \) \\
\addlinespace
\textbf{Inertial Coupling Ratios} & \( \beta_i \) & \( \beta_i = \frac{b_i}{M_1} \) \\
\addlinespace
\textbf{Damping Ratios} & \( \mathcal{N}_i \) & \( \mathcal{N}_i = \frac{C_i}{C_1} \) \\
 & \( \nu_i \) & \( \nu_i = \frac{c_i}{C_1} \) \\
\addlinespace
\textbf{Stiffness Ratios} & \( \Lambda_i \) & \( \Lambda_i = \frac{K_i}{K_1} \) \\
 & \( \lambda_i \) & \( \lambda_i = \frac{k_i}{K_1} \) \\
\addlinespace
\textbf{Decoupled Primary System} & \( \omega_{dc} \) & \( \omega_{dc} = \sqrt{\frac{K_1}{M_1}} \) \\
 & \( \zeta_{dc} \) & \( \zeta_{dc} = \frac{C_1}{2 M_1 \omega_{dc}} \) \\
\bottomrule
\end{tabular}
\end{table}

Using these parameters, the dimensionless equations of motion are expressed as:

\begin{equation}\label{Eq.EOM_dimensionless}
\mathbf{\bar{M}} \ddot{\mathbf{q}} + 2 \zeta_{dc} \omega_{dc} \mathbf{\bar{C}} \dot{\mathbf{q}} + \omega_{dc}^2 \mathbf{\bar{K}} \mathbf{q} = \mathbf{\bar{F}}(t)
\end{equation}

The dimensionless mass, damping, and stiffness matrices, along with the force vector, are defined in Equations \eqref{Eq.mass_matrix_dimensionless} to \eqref{Eq.force_vector_dimensionless}.

\paragraph{Dimensionless Mass Matrix}

\begin{equation}\label{Eq.mass_matrix_dimensionless}
\begin{aligned}
[\bar{M}] =& 
\begin{bmatrix}
\shortstack{$1 + \beta_1$ \\ $+ \beta_2 + \beta_3$} & 0 & \shortstack{$-\beta_1$ \\ \,} & \shortstack{$-\beta_2$ \\ \,} & \shortstack{$-\beta_3$ \\ \,} \\
0 & \shortstack{$\Gamma + \beta_4$ \\ $+ \beta_5 + \beta_6$} & \shortstack{$-\beta_4$ \\ \,} & \shortstack{$-\beta_5$ \\ \,} & \shortstack{$-\beta_6$ \\ \,} \\
-\beta_1 & -\beta_4 & \shortstack{$\mu_1 + \beta_1$ \\ $+ \beta_4 + \beta_7$ \\ $+ \beta_8 + \beta_9$ \\ $+ \beta_{10}$} & \shortstack{$-\beta_9$ \\ \,} & \shortstack{$-\beta_{10}$ \\ \,} \\
-\beta_2 & -\beta_5 & -\beta_9 & \shortstack{$\mu_2 + \beta_2$ \\ $+ \beta_5 + \beta_9$ \\ $+ \beta_{11} + \beta_{12}$} & \shortstack{$-\beta_{15}$ \\ \,} \\
-\beta_3 & -\beta_6 & -\beta_{10} & -\beta_{15} & \shortstack{$\mu_3 + \beta_3$ \\ $+ \beta_6 + \beta_{10}$ \\ $+ \beta_{13} + \beta_{14}$ \\ $+ \beta_{15}$}
\end{bmatrix}
\end{aligned}
\end{equation}

\paragraph{Dimensionless Damping Matrix}

\begin{equation}\label{Eq.damping_matrix_dimensionless}
\begin{aligned}
[\bar{C}] =& 
\begin{bmatrix}
\shortstack{$1 + \mathcal{N}_2$ \\ $+ \mathcal{N}_3 + \nu_1$ \\ $+ \nu_2 + \nu_3$} & -\mathcal{N}_3 & \shortstack{$-\nu_1$ \\ \,} & \shortstack{$-\nu_2$ \\ \,} & \shortstack{$-\nu_3$ \\ \,} \\
-\mathcal{N}_3 & \shortstack{$\mathcal{N}_3 + \mathcal{N}_4$ \\ $+ \mathcal{N}_5 + \nu_4$ \\ $+ \nu_5 + \nu_6$} & \shortstack{$-\nu_4$ \\ \,} & \shortstack{$-\nu_5$ \\ \,} & \shortstack{$-\nu_6$ \\ \,} \\
-\nu_1 & -\nu_2 & \shortstack{$\nu_1 + \nu_4$ \\ $+ \nu_7 + \nu_8$ \\ $+ \nu_9 + \nu_{10}$} & \shortstack{$-\nu_9$ \\ \,} & \shortstack{$-\nu_{10}$ \\ \,} \\
-\nu_2 & -\nu_5 & -\nu_9 & \shortstack{$\nu_2 + \nu_5$ \\ $+ \nu_9 + \nu_{11}$ \\ $+ \nu_{12} + \nu_{15}$} & \shortstack{$-\nu_{15}$ \\ \,} \\
-\nu_3 & -\nu_{6} & -\nu_{10} & -\nu_{15} & \shortstack{$\nu_3 + \nu_6$ \\ $+ \nu_{10} + \nu_{13}$ \\ $+ \nu_{14} + \nu_{15}$}
\end{bmatrix}
\end{aligned}
\end{equation}

\paragraph{Dimensionless Stiffness Matrix}

\begin{equation}\label{Eq.stiffness_matrix_dimensionless}
\begin{aligned}
[\bar{K}] =& 
\begin{bmatrix}
\shortstack{$1 + \Lambda_2$ \\ $+ \Lambda_3 + \lambda_1$ \\ $+ \lambda_2 + \lambda_3$} & -\Lambda_3 & \shortstack{$-\lambda_1$ \\ \,} & \shortstack{$-\lambda_2$ \\ \,} & \shortstack{$-\lambda_3$ \\ \,} \\
-\Lambda_3 & \shortstack{$\Lambda_3 + \Lambda_4$ \\ $+ \Lambda_5 + \lambda_4$ \\ $+ \lambda_5 + \lambda_6$} & \shortstack{$-\lambda_4$ \\ \,} & \shortstack{$-\lambda_5$ \\ \,} & \shortstack{$-\lambda_6$ \\ \,} \\
-\lambda_1 & -\lambda_2 & \shortstack{$\lambda_1 + \lambda_4$ \\ $+ \lambda_7 + \lambda_8$ \\ $+ \lambda_9 + \lambda_{10}$} & \shortstack{$-\lambda_9$ \\ \,} & \shortstack{$-\lambda_{10}$ \\ \,} \\
-\lambda_2 & -\lambda_5 & -\lambda_9 & \shortstack{$\lambda_2 + \lambda_5$ \\ $+ \lambda_9 + \lambda_{11}$ \\ $+ \lambda_{12} + \lambda_{15}$} & \shortstack{$-\lambda_{15}$ \\ \,} \\
-\lambda_3 & -\lambda_{6} & -\lambda_{10} & -\lambda_{15} & \shortstack{$\lambda_3 + \lambda_6$ \\ $+ \lambda_{10} + \lambda_{13}$ \\ $+ \lambda_{14} + \lambda_{15}$}
\end{bmatrix}
\end{aligned}
\end{equation}

\paragraph{Dimensionless Force Vector}

\begin{equation}\label{Eq.force_vector_dimensionless}
\begin{aligned}
[\bar{F}] = & \begin{bmatrix}
\frac{F_1(t)}{M_1} + 2 \zeta_{dc} \omega_{dc} (\dot{U}_{low} + \mathcal{N}_2 \dot{U}_{upp}) + \omega_{dc}^2 (U_{low} + \Lambda_2 U_{upp})\\
\frac{F_2(t)}{M_1} + 2 \zeta_{dc} \omega_{dc} (\mathcal{N}_4 \dot{U}_{low} + \mathcal{N}_5 \dot{U}_{upp}) + \omega_{dc}^2 (\Lambda_4 U_{low} + \Lambda_5 U_{upp}) \\
\beta_7 \ddot{U}_{low} +  \beta_8 \ddot{U}_{upp} +  2 \zeta_{dc} \omega_{dc} (\nu_7 \dot{U}_{low} + \nu_8 \dot{U}_{upp}) +\omega_{dc}^2 (\lambda_7 U_{low} + \lambda_8 U_{upp})\\
\beta_{11} \ddot{U}_{low} +  \beta_{12} \ddot{U}_{upp} +  2 \zeta_{dc} \omega_{dc} (\nu_{11} \dot{U}_{low} + \nu_{12} \dot{U}_{upp}) +\omega_{dc}^2 (\lambda_{11} U_{low} + \lambda_{12} U_{upp})\\
\beta_{13} \ddot{U}_{low} +  \beta_{14} \ddot{U}_{upp} +  2 \zeta_{dc} \omega_{dc} (\nu_{13} \dot{U}_{low} + \nu_{14} \dot{U}_{upp}) +\omega_{dc}^2 (\lambda_{13} U_{low} + \lambda_{14} U_{upp})
\end{bmatrix}
\end{aligned}
\end{equation}

\paragraph{Semi-Analytical Solutions for Harmonic Excitation}

The semi-analytical method assumes harmonic excitation and synchronized motion. The system response is expressed as:

\begin{align}\label{Eq.harmonic.solution.2dof3dof}
    \begin{bmatrix}
        U_1(t) \\
        U_2(t) \\
        u_1(t) \\
        u_2(t) \\
        u_3(t)
    \end{bmatrix} =
    \begin{bmatrix}
        A_1 \\
        A_2 \\
        a_1 \\
        a_2 \\
        a_3
    \end{bmatrix} e^{j \omega t}
\end{align}

where $A_1, A_2$ are the vibration amplitudes of the primary masses and $a_1, a_2, a_3$ are the amplitudes of the DVA masses.

The harmonic excitations are defined as:

\begin{align}\label{Eq.harmonic.excitations.2dof3dof}
    \begin{split}
        F_1(t) &= F_1 e^{j \omega t} \\
        F_2(t) &= F_2 e^{j \omega t} \\
        U_{Low}(t) &= A_{Low} e^{j \omega t} \\
        U_{Up}(t) &= A_{Up} e^{j \omega t}
    \end{split}
\end{align}

Substituting the harmonic solutions into the equations of motion yields the frequency domain formulation:

\begin{align}\label{Eq.frequency.domain.2dof3dof}
    \begin{bmatrix}
        A_1 \\
        A_2 \\
        a_1 \\
        a_2 \\
        a_3
    \end{bmatrix} = \omega_{dc}^2 \left( -\Omega^2 \mathbf{M} + j 2 \zeta_{dc} \Omega \mathbf{C} + \mathbf{K} \right)^{-1} \mathbf{F}
\end{align}

where $\mathbf{F}$ is the complex amplitude vector of the forcing function.



\subsection{Optimization Problem Formulation}

Building upon the complex 2DOF-3DOF mechanical system described in the previous section, the optimization problem is formulated to find optimal Dynamic Vibration Absorber (DVA) parameters that minimize deviations from desired system performance characteristics. The system comprises 48 independent design parameters distributed across 15 mass coupling coefficients ($\beta_1$ through $\beta_{15}$), 15 stiffness parameters ($\lambda_1$ through $\lambda_{15}$), 15 damping parameters ($\nu_1$ through $\nu_{15}$), and 3 DVA mass parameters ($\mu_1$, $\mu_2$, $\mu_3$).

The optimization problem can be mathematically stated as:

\begin{equation}\label{Eq.optimization_problem}
\begin{aligned}
\min_{\mathbf{x}} \quad & f(\mathbf{x}) = f_{primary}(\mathbf{x}) + f_{sparsity}(\mathbf{x}) + f_{error}(\mathbf{x}) \\
\text{subject to} \quad & \mathbf{x}_L \leq \mathbf{x} \leq \mathbf{x}_U \\
\end{aligned}
\end{equation}

where:
\begin{itemize}
    \item $\mathbf{x} \in \mathbb{R}^{48}$ is the design parameter vector
    \item $\mathbf{x}_L, \mathbf{x}_U \in \mathbb{R}^{48}$ are the lower and upper parameter bounds
    \item $f_{primary}(\mathbf{x})$, $f_{sparsity}(\mathbf{x})$, $f_{error}(\mathbf{x})$ are the objective function components
\end{itemize}

\subsubsection{Objective Function Definition}

The overall objective function is a weighted sum of three distinct components, each addressing different aspects of the optimization problem:

\begin{equation}\label{Eq.objective_function_detailed}
f(\mathbf{x}) = f_{primary}(\mathbf{x}) + f_{sparsity}(\mathbf{x}) + f_{error}(\mathbf{x})
\end{equation}

where each component is precisely defined and serves a specific purpose in the optimization process.

\textbf{Primary Objective Function ($f_{primary}(\mathbf{x})$):} The primary objective measures the deviation of the system's singular response from the optimal target value of 1.0:

\begin{equation}\label{Eq.primary_objective_detailed}
f_{primary}(\mathbf{x}) = \left| C_s(\mathbf{x}) - 1.0 \right|
\end{equation}

where the singular criteria $C_s(\mathbf{x})$ is computed as:

\begin{equation}\label{Eq.singular_response_detailed}
C_s(\mathbf{x}) = \sum_{i=1}^{5} CM_i(\mathbf{x})
\end{equation}

and $CM_i(\mathbf{x})$ is the composite measure for mass $i$:

\begin{equation}\label{Eq.composite_measure_detailed}
CM_i(\mathbf{x}) = \sum_{j} w_{ij} \cdot \frac{a_{ij}(\mathbf{x})}{t_{ij}}
\end{equation}

The composite measure integrates multiple performance criteria for each mass, where:
\begin{itemize}
    \item $w_{ij}$: Weight coefficient for criterion $j$ of mass $i$
    \item $a_{ij}(\mathbf{x})$: Actual performance value for criterion $j$ of mass $i$
    \item $t_{ij}$: Target performance value for criterion $j$ of mass $i$
\end{itemize}

The performance criteria include peak positions, peak values, bandwidths, slopes, and area under the curve, as extracted from the FRF analysis. The weights allow prioritization of different performance aspects, with typical values ranging from 0.05 to 1.0 depending on the importance of each criterion.

\textbf{Sparsity Penalty Function ($f_{sparsity}(\mathbf{x})$):} The sparsity penalty encourages solutions with smaller parameter magnitudes:

\begin{equation}\label{Eq.sparsity_penalty_detailed}
f_{sparsity}(\mathbf{x}) = \alpha \sum_{k=1}^{48} |x_k|
\end{equation}

where:
\begin{itemize}
    \item $\alpha$ is the sparsity weight coefficient 
    \item $x_k$ represents the $k$-th design parameter
    \item The L1 regularization promotes sparse solutions by penalizing non-zero parameter values
\end{itemize}

This function serves multiple purposes:

\begin{enumerate}
    \item \textbf{Regularization}: Prevents overfitting to specific frequency ranges by discouraging overly complex parameter combinations
    \item \textbf{Practical Implementation}: Encourages simpler DVA configurations that are easier to manufacture and maintain
    \item \textbf{Robustness Enhancement}: Reduces sensitivity to parameter variations in real-world applications
\end{enumerate}

\textbf{Percentage Error Component ($f_{error}(\mathbf{x})$):} The percentage error component captures detailed performance deviations:

\begin{align}\label{Eq.percentage_error_detailed}
f_{error}(\mathbf{x}) &= \frac{1}{\gamma} \sum_{i} \sum_{j} \left| PD_{ij}(\mathbf{x}) \right|\\
PD_{ij}(\mathbf{x}) &= \left( \frac{a_{ij}(\mathbf{x}) - t_{ij}}{|t_{ij}|} \right) \times 100\%
\end{align}
where:
\begin{itemize}
    \item $\gamma$ is the scaling factor (default: 1000)
    \item $PD_{ij}(\mathbf{x})$ is the percentage difference for criterion $j$ of mass $i$
    \item The absolute value prevents cancellation between positive and negative errors
\end{itemize}

This component ensures comprehensive evaluation by:
\begin{itemize}
    \item Capturing all performance criteria deviations in the objective function
    \item Enabling consideration of both primary response characteristics and detailed performance metrics
    \item Allowing different criteria to have varying importance through target value specification
    \item Providing a normalized error measure that can be compared across different criteria types
    \item Balancing the contribution of detailed metrics with the primary objective through the scaling factor $\gamma$
\end{itemize}

The scaling factor $\gamma$ plays a crucial role in balancing the contribution of the percentage error component with the other objectives. A larger $\gamma$ reduces the influence of detailed percentage errors, while a smaller $\gamma$ increases their importance in the overall optimization.

\subsubsection{Performance Criteria Hierarchy}
The multi-objective framework evaluates performance across several hierarchical categories:

\begin{enumerate}
    \item \textbf{Modal Performance Criteria:}
    \begin{itemize}
        \item Peak positions ($\omega_{peak,i}$): Target resonant frequencies for each mass
        \item Peak values ($A_{peak,i}$): Target response amplitudes at resonant frequencies
        \item Critical for modal alignment and vibration control effectiveness
    \end{itemize}
    

    \item \textbf{Inter-Modal Criteria:}
    \begin{itemize}
        \item Bandwidths ($\Delta\omega_{i,j}$): Target frequency ranges between resonances
        \item Slopes ($s_{i,j}$): Rate of change between peaks
        \item Affects system stability and control bandwidth
    \end{itemize}

    \item \textbf{Global Response Criteria:}
    \begin{itemize}
        \item Area under curve: Total response energy across frequency range
        \item Maximum slope: Maximum rate of amplitude change
        \item Provides overall system response characteristics
    \end{itemize}
\end{enumerate}

\textbf{Weight Assignment Strategy:} The assignment of weights to each objective function is a critical aspect of the optimization process, as it directly influences the balance between different performance criteria in the total fitness function. In this methodology, the program calculates $C_s - 1$ as one of the objective functions, and it is important to ensure that the contribution of each objective is properly normalized.

To achieve a meaningful and interpretable fitness value, it is recommended that the sum of all weights assigned to the objectives does not exceed 1, and ideally, all weights should sum exactly to 1. This normalization ensures that each objective's influence is proportional and that the overall fitness function is calculated correctly, preventing any single criterion from dominating the optimization process due to disproportionate weighting.

The general structure for assigning weights is as follows:

\begin{equation}\label{Eq.weight_hierarchy_detailed}
w_{ij} = w_i \cdot w_{base,j}
\end{equation}

where:
\begin{itemize}
    \item $w_i$: Weight associated with a specific mass or subsystem (often set to 1.0 for uniform treatment, but can be adjusted for prioritization)
    \item $w_{base,j}$: Base weight assigned to each criterion type $j$ (e.g., peak position, peak value, bandwidth, slope, area, etc.)
\end{itemize}

The sum of all $w_{ij}$ across all masses and criteria should satisfy:
\begin{equation}
\sum_{i,j} w_{ij} \leq 1
\end{equation}
and, for best normalization, it is preferable to enforce
\begin{equation}
\sum_{i,j} w_{ij} = 1
\end{equation}

This approach ensures that the fitness function remains consistent and interpretable, especially when combining $C_s - 1$ with other objectives. However, since the program is open-source, users have the flexibility to modify the weighting scheme as needed for their specific application or research focus. Adjusting the weights allows for custom prioritization of objectives, but care should be taken to maintain the normalization condition for optimal performance and comparability of results.

\textbf{Objective Function Interactions:} 

In multi-objective optimization, especially in the context of genetic algorithms for engineering design, the total objective function $f_{total}(\mathbf{x})$ is typically composed of several distinct terms, each representing a different aspect of system performance or a different design goal. These terms may include, for example, a primary performance objective (such as minimizing vibration amplitude), a sparsity-promoting penalty (to encourage simpler or more efficient designs), and an error or constraint penalty (to penalize infeasible or undesirable solutions).


\subsubsection{Fitness Evaluation through FRF Analysis}


\textbf{FRF Mathematical Foundation:} The FRF analysis is based on the frequency domain representation of the system dynamics:

\begin{equation}\label{Eq.frequency_domain_system_detailed}
\mathbf{X}(\omega) = \mathbf{G}(\omega) \mathbf{F}(\omega)
\end{equation}

where:
\begin{itemize}
    \item $\mathbf{X}(\omega) \in \mathbb{C}^{5}$: Displacementresponse vector (5 DOF)
    \item $\mathbf{G}(\omega) \in \mathbb{C}^{5 \times 5}$: Frequency response function matrix
    \item $\mathbf{F}(\omega) \in \mathbb{C}^{5}$: Forcing function vector including external and base excitations
\end{itemize}

The frequency response function matrix is computed as:

\begin{equation}\label{Eq.frf_matrix_detailed}
\mathbf{G}(\omega) = \left[ -\omega^2 \mathbf{M} + j\omega \mathbf{C} + \mathbf{K} \right]^{-1}
\end{equation}

where $\mathbf{M}$, $\mathbf{C}$, $\mathbf{K}$ are the system matrices defined in the mechanical system formulation.


\textbf{Performance Metric Calculations:} Each mass response undergoes comprehensive analysis:

\begin{enumerate}
    \item \textbf{Peak Frequency (X-Value) Analysis:}
    \begin{equation}\label{Eq.peak_frequency_analysis_detailed}
    \omega_{peak,i} = \omega_{j^*} \quad \text{where} \quad j^* = \arg\max_{j} |X_i(\omega_j)|, \quad \forall i = 1,\ldots,5
    \end{equation}

    \item \textbf{Peak Amplitude (Y-Value) Analysis:}
    \begin{equation}\label{Eq.peak_amplitude_analysis_detailed}
    A_{peak,i} = |X_i(\omega_{peak,i})| \quad \forall i = 1,\ldots,5
    \end{equation}

    \item \textbf{Bandwidth Analysis:}
    \begin{equation}\label{Eq.bandwidth_analysis_detailed}
    \Delta\omega_{i,j} = |\omega_{peak,j} - \omega_{peak,i}| \quad \forall i < j
    \end{equation}

    \item \textbf{Slope Analysis:}
    \begin{equation}\label{Eq.slope_analysis_detailed}
    s_{i,j} = \frac{A_{peak,j} - A_{peak,i}}{\omega_{peak,j} - \omega_{peak,i}} \quad \forall i < j
    \end{equation}

    \item \textbf{Area Analysis:}
    \begin{equation}\label{Eq.area_analysis_detailed}
    Area_i = \int_{\omega_{start}}^{\omega_{end}} |X_i(\omega)| d\omega \quad \forall i = 1,\ldots,5
\end{equation}
\end{enumerate}

The area integral is computed using Simpson's rule for numerical accuracy:

\begin{equation}\label{Eq.area_simpson_detailed}
Area_i \approx \frac{h}{3} \left[|X_i(\omega_0)| + 4\sum_{k=1}^{n/2} |X_i(\omega_{2k-1})| + 2\sum_{k=1}^{n/2-1} |X_i(\omega_{2k})| + |X_i(\omega_n)| \right]
\end{equation}

where $h$ is the frequency step size and $n$ is the number of frequency points.


\section{Input Parameters}
\label{sec:input-parameters}

This section presents the comprehensive input parameter configuration used in the \softwareName{} optimization framework. The parameters are organized into several categories that define the mechanical system characteristics, optimization objectives, algorithm settings, and design constraints.

\subsection{Main System Parameters}

The main system configuration defines the fundamental characteristics of the 2DOF primary structure and its base excitations. Table~\ref{Tab:main_system_parameters} presents the dimensionless parameters that characterize the primary system dynamics.

\begin{table}[h!]
\centering
\caption{Main System Configuration Parameters}
\label{Tab:main_system_parameters}
\begin{tabular}{lcc}
\toprule
\textbf{Parameter} & \textbf{Value} & \textbf{Description} \\
\midrule
$\mu$ & 1.0 & Mass ratio ($M_2/M_1$) \\
$\Lambda_1$ & 1.0 & Primary stiffness ratio ($K_1/K_1$) \\
$\Lambda_2$ & 1.0 & Secondary stiffness ratio ($K_2/K_1$) \\
$\Lambda_3$ & 0.5 & Tertiary stiffness ratio ($K_3/K_1$) \\
$\Lambda_4$ & 0.5 & Quaternary stiffness ratio ($K_4/K_1$) \\
$\Lambda_5$ & 0.5 & Quinary stiffness ratio ($K_5/K_1$) \\
$\mathcal{N}_1$ & 0.75 & Primary damping ratio ($C_1/C_1$) \\
$\mathcal{N}_2$ & 0.75 & Secondary damping ratio ($C_2/C_1$) \\
$\mathcal{N}_3$ & 0.75 & Tertiary damping ratio ($C_3/C_1$) \\
$\mathcal{N}_4$ & 0.75 & Quaternary damping ratio ($C_4/C_1$) \\
$\mathcal{N}_5$ & 0.75 & Quinary damping ratio ($C_5/C_1$) \\
$a_{low}$ & 0.05 & Lower base excitation amplitude \\
$a_{up}$ & 0.05 & Upper base excitation amplitude \\
$f_1$ & 100.0 & External force amplitude on mass 1 \\
$f_2$ & 100.0 & External force amplitude on mass 2 \\
$\omega_{dc}$ & 100.0 & Decoupled natural frequency (rad/s) \\
$\zeta_{dc}$ & 0.01 & Decoupled damping ratio \\
\bottomrule
\end{tabular}
\end{table}

\subsection{Dynamic Vibration Absorber Initial Configuration}

The DVA system is initially configured with all coupling parameters set to zero, representing an uncoupled state. Table~\ref{Tab:dva_initial_config} shows the initial DVA parameter values.

\begin{table}[h!]
\centering
\caption{Initial DVA Configuration (All Parameters Set to Zero)}
\label{Tab:dva_initial_config}
\begin{tabular}{lcc}
\toprule
\textbf{Parameter Group} & \textbf{Count} & \textbf{Initial Value} \\
\midrule
$\beta_i$ (Mass coupling) & 15 & 0.0 \\
$\lambda_i$ (Stiffness coupling) & 15 & 0.0 \\
$\mu_i$ (DVA masses) & 3 & 0.0 \\
$\nu_i$ (Damping coupling) & 15 & 0.0 \\
\bottomrule
\end{tabular}
\end{table}

This initial configuration represents a baseline state where the primary 2DOF system operates without any DVA coupling, providing a reference point for optimization performance assessment.

\subsection{Target Performance Specifications}

The optimization targets focus on achieving specific frequency response characteristics for each mass in the system. Table~\ref{Tab:target_specifications} presents the target specifications for mass 1, which is the primary focus of the optimization.

\begin{table}[h!]
\centering
\caption{Target Performance Specifications for Mass 1}
\label{Tab:target_specifications}
\begin{tabular}{lcc}
\toprule
\textbf{Performance Criterion} & \textbf{Target Value} & \textbf{Weight} \\
\midrule
Peak Position 2 & 100.0 rad/s & 0.5 \\
Peak Position 3 & 250.0 rad/s & 0.5 \\
All other criteria & 0.0 & 0.0 \\
\bottomrule
\end{tabular}
\end{table}

The target configuration emphasizes achieving specific resonant frequencies at 100 rad/s and 250 rad/s for mass 1, with equal weighting for both targets. All other performance criteria (peak values, bandwidths, slopes, and area under curve) are set to zero weight, indicating they are not considered in this optimization scenario.

\subsection{Frequency Analysis Configuration}

The frequency response analysis is configured to cover a comprehensive range of excitation frequencies. Table~\ref{Tab:frequency_config} presents the frequency analysis parameters.

\begin{table}[h!]
\centering
\caption{Frequency Response Analysis Configuration}
\label{Tab:frequency_config}
\begin{tabular}{lcc}
\toprule
\textbf{Parameter} & \textbf{Value} & \textbf{Description} \\
\midrule
$\omega_{start}$ & 0.0 rad/s & Starting frequency \\
$\omega_{end}$ & 350.0 rad/s & Ending frequency \\
$\omega_{points}$ & 350 & Number of frequency points \\
Plot figure & True & Enable response plotting \\
Show peaks & False & Disable peak marking \\
Show slopes & False & Disable slope marking \\
Interpolation method & None & No interpolation \\
Interpolation points & 1000 & Reserved for future use \\
\bottomrule
\end{tabular}
\end{table}

The frequency range from 0 to 350 rad/s encompasses the target frequencies (100 and 250 rad/s) with sufficient resolution for accurate peak detection and analysis.

\subsection{Genetic Algorithm Optimization Settings}

The genetic algorithm is configured with comprehensive parameters for population management, genetic operators, and convergence criteria. Table~\ref{Tab:ga_settings} presents the main GA configuration parameters.

\begin{table}[h!]
\centering
\caption{Genetic Algorithm Configuration Parameters}
\label{Tab:ga_settings}
\begin{tabular}{lcc}
\toprule
\textbf{Parameter Category} & \textbf{Parameter} & \textbf{Value} \\
\midrule
\multirow{3}{*}{Population} & Size & 150 \\
& Minimum & 100 \\
& Maximum & 200 \\
\midrule
\multirow{2}{*}{Evolution} & Generations & 150 \\
& Tolerance & 0.1 \\
\midrule
\multirow{2}{*}{Genetic Operators} & Crossover probability & 0.7 \\
& Mutation probability & 0.2 \\
\midrule
\multirow{2}{*}{Objective Function} & Sparsity weight ($\alpha$) & 0.02 \\
& Error scale ($\gamma$) & 1000.0 \\
\midrule
Benchmarking & Number of runs & 10 \\
\bottomrule
\end{tabular}
\end{table}

\subsection{Advanced Algorithm Features}

The configuration includes several advanced features that can be enabled or disabled based on optimization requirements. Table~\ref{Tab:advanced_features} summarizes the status of these features.

\begin{table}[h!]
\centering
\caption{Advanced Algorithm Features Configuration}
\label{Tab:advanced_features}
\begin{tabular}{lcc}
\toprule
\textbf{Feature} & \textbf{Status} & \textbf{Description} \\
\midrule
Adaptive Rates & Disabled & Fixed genetic operator probabilities \\
ML-Bandit Controller & Disabled & Machine learning parameter adaptation \\
RL Controller & Disabled & Reinforcement learning control \\
Surrogate Screening & Disabled & KNN-based fitness prediction \\
Neural Seeding & Disabled & Neural network-based population initialization \\
\bottomrule
\end{tabular}
\end{table}

\subsection{Parameter Bounds and Constraints}

The optimization problem includes 48 design parameters with comprehensive bounds and constraints. Table~\ref{Tab:parameter_bounds} presents the parameter bounds configuration.

\begin{table}[h!]
\centering
\caption{Design Parameter Bounds and Constraints}
\label{Tab:parameter_bounds}
\begin{tabular}{lccc}
\toprule
\textbf{Parameter Group} & \textbf{Count} & \textbf{Bounds} & \textbf{Fixed Parameters} \\
\midrule
$\beta_i$ (Mass coupling) & 15 & [0.0, 1.0] & None \\
$\lambda_i$ (Stiffness coupling) & 15 & [0.0, 1.0] & None \\
$\mu_i$ (DVA masses) & 3 & [0.0, 1.0] & None \\
$\nu_i$ (Damping coupling) & 14 & [0.0, 1.0] & $\nu_{15} = 0.0$ \\
\bottomrule
\end{tabular}
\end{table}

All parameters are bounded between 0.0 and 1.0 in dimensionless form, with only one parameter ($\nu_{15}$) fixed at 0.0. This configuration provides a total of 47 free parameters for optimization.

\subsection{Omega Sensitivity Analysis Configuration}

The system includes configuration for frequency sensitivity analysis, which can be used to assess the robustness of optimized solutions. Table~\ref{Tab:omega_sensitivity} presents these settings.

\begin{table}[h!]
\centering
\caption{Omega Sensitivity Analysis Configuration}
\label{Tab:omega_sensitivity}
\begin{tabular}{lcc}
\toprule
\textbf{Parameter} & \textbf{Value} & \textbf{Description} \\
\midrule
Initial points & 100 & Starting number of evaluation points \\
Maximum points & 2000 & Maximum allowed evaluation points \\
Step size & 1000 & Points added per iteration \\
Threshold & 0.01 & Convergence threshold \\
Maximum iterations & 200 & Maximum optimization iterations \\
Target mass & Mass 1 & Primary mass for sensitivity analysis \\
Plot results & True & Enable sensitivity plotting \\
Use optimal & True & Use optimal parameters as reference \\
\bottomrule
\end{tabular}
\end{table}

\subsection{Configuration Summary and Implications}

The input parameter configuration defines a comprehensive optimization problem with the following key characteristics:

\begin{enumerate}
    \item \textbf{System Complexity}: 2DOF primary system with 3 DVA masses, totaling 5 degrees of freedom
    \item \textbf{Design Space}: 47 free parameters with normalized bounds [0, 1]
    \item \textbf{Optimization Focus}: Frequency response targeting at 100 and 250 rad/s for mass 1
    \item \textbf{Algorithm Configuration}: Traditional GA with population size 150 and 150 generations
    \item \textbf{Advanced Features}: All advanced controllers and surrogates disabled for baseline comparison
    \item \textbf{Benchmarking}: 10 independent runs for statistical analysis
\end{enumerate}

This configuration provides a robust foundation for evaluating the performance of different optimization strategies and serves as a baseline for comparing advanced features such as adaptive controllers, surrogate models, and intelligent seeding methods.

\section{Particle Swarm Optimization Methodology}
\label{sec:pso_methodology}

\subsection{Introduction to Particle Swarm Optimization}

Particle Swarm Optimization (PSO) is a population-based stochastic optimization technique inspired by the social behavior of bird flocking or fish schooling. In PSO, a swarm of particles moves through the search space, guided by their own best known position and the swarm's best known position. This methodology has been successfully applied to various engineering optimization problems, including the design of Dynamic Vibration Absorbers (DVAs).

\subsection{Mathematical Foundation of PSO}

\subsubsection{Basic PSO Algorithm}

The fundamental PSO algorithm operates on a swarm of $N$ particles, where each particle $i$ at iteration $t$ has:
\begin{itemize}
    \item A position vector $\mathbf{x}_i(t) \in \mathbb{R}^P$ representing a potential solution
    \item A velocity vector $\mathbf{v}_i(t) \in \mathbb{R}^P$ determining the particle's movement direction and speed
    \item A personal best position $\mathbf{p}_i(t)$ representing the best position this particle has encountered
    \item A neighborhood best position $\mathbf{g}_i(t)$ representing the best position in the particle's neighborhood
\end{itemize}

The position and velocity update equations form the core of the PSO algorithm:

\begin{align}
\mathbf{v}_i(t+1) &= w \cdot \mathbf{v}_i(t) + c_1 \cdot r_1 \cdot (\mathbf{p}_i(t) - \mathbf{x}_i(t)) + c_2 \cdot r_2 \cdot (\mathbf{g}_i(t) - \mathbf{x}_i(t)) \label{eq:pso_velocity_update} \\
\mathbf{x}_i(t+1) &= \mathbf{x}_i(t) + \mathbf{v}_i(t+1) \label{eq:pso_position_update}
\end{align}

where:
\begin{itemize}
    \item $w$ is the inertia weight controlling the influence of previous velocity
    \item $c_1$ is the cognitive coefficient controlling attraction to personal best
    \item $c_2$ is the social coefficient controlling attraction to neighborhood best
    \item $r_1, r_2 \sim \text{Uniform}(0,1)$ are random numbers for stochasticity
\end{itemize}

\subsubsection{Constriction Factor for Stability}

To ensure convergence and prevent velocity explosion, the constriction factor approach is implemented:

\begin{equation}
\chi = \frac{2}{|2 - \phi - \sqrt{\phi^2 - 4\phi}|} \label{eq:constriction_factor}
\end{equation}

where $\phi = c_1 + c_2$. When $\phi > 4$, the constriction factor ensures guaranteed convergence. The velocity update becomes:

\begin{equation}
\mathbf{v}_i(t+1) = \chi \cdot [w \cdot \mathbf{v}_i(t) + c_1 \cdot r_1 \cdot (\mathbf{p}_i(t) - \mathbf{x}_i(t)) + c_2 \cdot r_2 \cdot (\mathbf{g}_i(t) - \mathbf{x}_i(t))] \label{eq:pso_constricted_velocity}
\end{equation}

\subsection{Advanced PSO Features in DeVana}

The DeVana implementation extends the basic PSO with numerous advanced features specifically designed for DVA optimization problems.

\subsubsection{Adaptive Parameter Control}

\paragraph{Adaptive Inertia Weight}
The inertia weight is adaptively adjusted based on multiple factors to balance exploration and exploitation:

\begin{equation}
w(t) = w_{\text{linear}}(t) + w_{\text{nonlinear}}(t) + w_{\text{fitness}}(t) + w_{\text{diversity}}(t) \label{eq:adaptive_inertia}
\end{equation}

where each component addresses different aspects:

\textbf{Linear Time-Varying Component:}
\begin{equation}
w_{\text{linear}}(t) = w_{\max} - (w_{\max} - w_{\min}) \cdot \frac{t}{T-1} \label{eq:linear_inertia}
\end{equation}

\textbf{Nonlinear Time-Varying Component:}
\begin{equation}
w_{\text{nonlinear}}(t) = w_{\max} - (w_{\max} - w_{\min}) \cdot \left(\frac{t}{T-1}\right)^n \label{eq:nonlinear_inertia}
\end{equation}
where $n = 1.2$ provides faster initial decrease.

\textbf{Fitness-Based Component:}
\begin{equation}
w_{\text{fitness}}(t) = 0.5 \cdot \left(1 - \frac{f_{\text{best}}(t)}{f_{\text{avg}}(t)}\right) \label{eq:fitness_inertia}
\end{equation}

\textbf{Diversity-Based Component:}
\begin{equation}
w_{\text{diversity}}(t) = w_{\min} + (1 - \min(1, \frac{D(t)}{D_{\text{threshold}}})) \cdot (w_{\max} - w_{\min}) \cdot 0.5 \label{eq:diversity_inertia}
\end{equation}

\paragraph{Adaptive Acceleration Coefficients}
The cognitive and social coefficients are adaptively adjusted:

\begin{align}
c_1(t) &= c_{1,\max} - (c_{1,\max} - c_{1,\min}) \cdot \frac{t}{T-1} + \Delta c_1^{\text{diversity}}(t) \label{eq:adaptive_c1} \\
c_2(t) &= c_{2,\min} + (c_{2,\max} - c_{2,\min}) \cdot \frac{t}{T-1} + \Delta c_2^{\text{diversity}}(t) \label{eq:adaptive_c2}
\end{align}

where the diversity-based adjustments are:
\begin{align}
\Delta c_1^{\text{diversity}}(t) &= \begin{cases}
(c_{1,\max} - c_{1,\min}) \cdot (1 - \frac{D(t)}{D_{\text{threshold}}}) \cdot 0.5 & \text{if } D(t) < D_{\text{threshold}} \\
0 & \text{otherwise}
\end{cases} \label{eq:diversity_c1} \\
\Delta c_2^{\text{diversity}}(t) &= \begin{cases}
-(c_{2,\max} - c_{2,\min}) \cdot (1 - \frac{D(t)}{D_{\text{threshold}}}) \cdot 0.5 & \text{if } D(t) < D_{\text{threshold}} \\
0 & \text{otherwise}
\end{cases} \label{eq:diversity_c2}
\end{align}

\subsubsection{Neighborhood Topologies}

The implementation supports multiple neighborhood topologies that affect information flow through the swarm:

\paragraph{Global Topology}
All particles are connected to each other, providing fastest convergence but potential for local optima trapping.

\paragraph{Ring Topology}
Each particle is connected to its immediate neighbors:
\begin{equation}
\mathcal{N}_i^{\text{ring}} = \{i-1, i, i+1\} \pmod{N} \label{eq:ring_neighborhood}
\end{equation}

\paragraph{Von Neumann Topology}
Grid-like neighborhood structure with 4 neighbors per particle:
\begin{equation}
\mathcal{N}_i^{\text{von}} = \{i, \text{north}(i), \text{south}(i), \text{east}(i), \text{west}(i)\} \label{eq:von_neumann_neighborhood}
\end{equation}

\paragraph{Random Topology}
Random connections that change periodically:
\begin{equation}
\mathcal{N}_i^{\text{random}} = \{i\} \cup \text{RandomSample}(\{1,\ldots,N\} \setminus \{i\}, k) \label{eq:random_neighborhood}
\end{equation}

\subsubsection{Diversity Maintenance}

\paragraph{Swarm Diversity Calculation}
Diversity is measured as the normalized average distance to the swarm centroid:

\begin{equation}
D(t) = \frac{1}{N \cdot \sqrt{P}} \sum_{i=1}^{N} \sqrt{\sum_{j=1}^{P} \left(\frac{x_{i,j}(t) - \bar{x}_j(t)}{\text{range}_j}\right)^2} \label{eq:diversity_measure}
\end{equation}

where $\bar{x}_j(t)$ is the centroid of dimension $j$ and $\text{range}_j$ is the parameter range.

\paragraph{Mutation for Diversity Enhancement}
When diversity drops below threshold, Gaussian mutation is applied:

\begin{equation}
x_{i,j}^{\text{mutated}}(t) = x_{i,j}(t) + \mathcal{N}(0, \sigma_j^2) \label{eq:pso_mutation}
\end{equation}

where $\sigma_j = 0.1 \cdot \text{range}_j$ is the mutation strength.

\subsubsection{Boundary Handling Strategies}

\paragraph{Absorbing Boundaries}
Particles hitting boundaries are stopped:
\begin{equation}
x_{i,j}(t+1) = \begin{cases}
x_j^{\min} & \text{if } x_{i,j}(t+1) < x_j^{\min} \\
x_j^{\max} & \text{if } x_{i,j}(t+1) > x_j^{\max} \\
x_{i,j}(t+1) & \text{otherwise}
\end{cases} \label{eq:absorbing_boundaries}
\end{equation}

\paragraph{Reflecting Boundaries}
Particles bounce off boundaries:
\begin{equation}
x_{i,j}(t+1) = \begin{cases}
x_j^{\min} + |x_j^{\min} - x_{i,j}(t+1)| & \text{if } x_{i,j}(t+1) < x_j^{\min} \\
x_j^{\max} - |x_{i,j}(t+1) - x_j^{\max}| & \text{if } x_{i,j}(t+1) > x_j^{\max} \\
x_{i,j}(t+1) & \text{otherwise}
\end{cases} \label{eq:reflecting_boundaries}
\end{equation}

\paragraph{Invisible Boundaries}
Particles move freely but receive penalty in fitness evaluation:
\begin{equation}
\text{Penalty} = \sum_{j=1}^{P} \begin{cases}
\left(\frac{x_j^{\min} - x_{i,j}}{x_j^{\max} - x_j^{\min}}\right)^2 & \text{if } x_{i,j} < x_j^{\min} \\
\left(\frac{x_{i,j} - x_j^{\max}}{x_j^{\max} - x_j^{\min}}\right)^2 & \text{if } x_{i,j} > x_j^{\max} \\
0 & \text{otherwise}
\end{cases} \label{eq:invisible_boundaries}
\end{equation}

\subsubsection{Velocity Clamping}

To prevent velocity explosion, velocities are clamped to a maximum value:

\begin{equation}
v_{i,j}(t+1) = \max(-v_{j}^{\max}, \min(v_{j}^{\max}, v_{i,j}(t+1))) \label{eq:velocity_clamping}
\end{equation}

where $v_{j}^{\max} = 0.1 \cdot (x_j^{\max} - x_j^{\min})$ is the maximum velocity for dimension $j$.

\subsubsection{Stagnation Detection and Recovery}

\paragraph{Personal Stagnation}
Each particle tracks iterations without improvement:
\begin{equation}
s_i(t+1) = \begin{cases}
0 & \text{if } f(\mathbf{x}_i(t+1)) < f(\mathbf{p}_i(t)) \\
s_i(t) + 1 & \text{otherwise}
\end{cases} \label{eq:personal_stagnation}
\end{equation}

\paragraph{Reinitialization Strategy}
Stagnant particles are reinitialized around the global best:
\begin{equation}
x_{i,j}^{\text{new}} = x_{j}^{\text{gb}} + \mathcal{U}(-\text{radius}_j, \text{radius}_j) \label{eq:reinitialization}
\end{equation}

where $\text{radius}_j = 0.1 \cdot (x_j^{\max} - x_j^{\min})$ and $x_{j}^{\text{gb}}$ is the global best position.

\subsubsection{Quasi-Random Initialization}

For better search space coverage, Sobol sequences are used for initialization:

\begin{equation}
x_{i,j}(0) = x_j^{\min} + z_{i,j} \cdot (x_j^{\max} - x_j^{\min}) \label{eq:sobol_initialization}
\end{equation}

where $z_{i,j}$ are Sobol sequence values in $[0,1]$.

\subsection{Enhanced PSO Algorithm with Advanced Features}

\begin{algorithm}[H]
\caption{Enhanced PSO Algorithm for DVA Optimization}
\begin{algorithmic}[1]
\REQUIRE Parameter bounds $[x_j^{\min}, x_j^{\max}]$, fixed parameters, swarm size $N$, max iterations $T$, tolerance $\tau$
\STATE Initialize swarm using Sobol sequences or random initialization
\STATE Set up neighborhood topology (Global/Ring/Von Neumann/Random)
\STATE Initialize personal and global best positions
\FOR{$t = 1$ to $T$}
    \STATE Update adaptive parameters: $w(t)$, $c_1(t)$, $c_2(t)$
    \STATE Calculate swarm diversity $D(t)$
    \FOR{each particle $i$}
        \STATE Update velocity using Eq.~\eqref{eq:pso_constricted_velocity}
        \STATE Apply velocity clamping using Eq.~\eqref{eq:velocity_clamping}
        \STATE Update position using Eq.~\eqref{eq:pso_position_update}
        \STATE Handle boundary violations using selected strategy
        \IF{$D(t) < D_{\text{threshold}}$}
            \STATE Apply mutation using Eq.~\eqref{eq:pso_mutation}
        \ENDIF
        \STATE Evaluate fitness $f(\mathbf{x}_i(t+1))$
        \STATE Update personal best if improved
        \STATE Update stagnation counter using Eq.~\eqref{eq:personal_stagnation}
        \IF{$s_i(t+1) \geq s_{\text{limit}}$}
            \STATE Reinitialize particle around global best using Eq.~\eqref{eq:reinitialization}
        \ENDIF
    \ENDFOR
    \STATE Update neighborhood best positions
    \STATE Update global best position
    \IF{$f(\mathbf{x}^{\text{gb}}) \leq \tau$}
        \STATE \textbf{break} (convergence reached)
    \ENDIF
    \IF{early stopping conditions met}
        \STATE \textbf{break} (early stopping)
    \ENDIF
\ENDFOR
\STATE \textbf{return} global best position and fitness
\end{algorithmic}
\end{algorithm}

\subsection{Machine Learning and Reinforcement Learning Controllers}

\subsubsection{ML-Bandit Controller}

The ML-bandit controller adapts PSO parameters using Upper Confidence Bound (UCB) selection:

\begin{equation}
a_t = \arg\max_{a \in \mathcal{A}} \left[ \hat{R}_t(a) + c \sqrt{\frac{\ln t}{N_t(a)}} \right] \label{eq:ml_bandit_selection}
\end{equation}

where $\mathcal{A}$ is the action space of parameter adjustments, $\hat{R}_t(a)$ is the average reward for action $a$, and $N_t(a)$ is the number of times action $a$ has been selected.

\paragraph{Reward Function}
The reward balances improvement, speed, and diversity:

\begin{equation}
R_t = \frac{\max\{0, f_{t-1}^{\star} - f_t^{\star}\}}{\max\{\epsilon, T_t\}} \cdot \frac{1}{\max\{1, E_t\}} - w_{\text{div}} \cdot |CV_t - CV^{\text{target}}| \label{eq:ml_bandit_reward}
\end{equation}

where $CV_t = \frac{\sigma_t}{|\mu_t| + \epsilon}$ is the coefficient of variation.

\subsubsection{Reinforcement Learning Controller}

The RL controller uses Q-learning with $\epsilon$-greedy policy:

\begin{equation}
Q(s_t, a_t) \leftarrow Q(s_t, a_t) + \alpha [R_t + \gamma \max_{a'} Q(s_{t+1}, a') - Q(s_t, a_t)] \label{eq:rl_q_update}
\end{equation}

where $\alpha$ is the learning rate, $\gamma$ is the discount factor, and $\epsilon$ decays over time.

\subsection{Fitness Evaluation and Objective Function}

The PSO fitness function is identical to the GA objective function (Eq.~\eqref{Eq.objective_function_detailed}):

\begin{equation}
f(\mathbf{x}) = f_{\text{primary}}(\mathbf{x}) + f_{\text{sparsity}}(\mathbf{x}) + f_{\text{error}}(\mathbf{x}) \label{eq:pso_fitness}
\end{equation}

where:
\begin{align}
f_{\text{primary}}(\mathbf{x}) &= |C_s(\mathbf{x}) - 1.0| \label{eq:pso_primary} \\
f_{\text{sparsity}}(\mathbf{x}) &= \alpha \sum_{k=1}^{P} |x_k| \label{eq:pso_sparsity} \\
f_{\text{error}}(\mathbf{x}) &= \frac{1}{\gamma} \sum_{i,j} |PD_{ij}(\mathbf{x})| \label{eq:pso_error}
\end{align}

\subsection{Software Implementation Architecture}

\subsubsection{PSOWorker Class Structure}

The PSOWorker class implements the enhanced PSO algorithm as a QThread for background execution:

\begin{itemize}
    \item \textbf{Initialization}: Sets up swarm, neighborhoods, and adaptive parameters
    \item \textbf{Main Loop}: Implements the core PSO algorithm with all advanced features
    \item \textbf{Evaluation}: Integrates with the FRF analysis pipeline
    \item \textbf{Signaling}: Communicates progress and results to the GUI
    \item \textbf{Metrics Collection}: Tracks detailed performance metrics
\end{itemize}

\subsubsection{Key Methods and Features}

\textbf{Adaptive Parameter Methods:}
\begin{itemize}
    \item \texttt{adaptive\_inertia\_weight()}: Implements Eq.~\eqref{eq:adaptive_inertia}
    \item \texttt{adaptive\_acceleration\_coefficients()}: Implements Eqs.~\eqref{eq:adaptive_c1}--\eqref{eq:adaptive_c2}
    \item \texttt{calculate\_diversity()}: Implements Eq.~\eqref{eq:diversity_measure}
\end{itemize}

\textbf{Topology and Neighborhood Methods:}
\begin{itemize}
    \item \texttt{create\_neighborhoods()}: Implements various topology types
    \item \texttt{update\_neighborhoods()}: Handles dynamic topology updates
\end{itemize}

\textbf{Boundary and Constraint Handling:}
\begin{itemize}
    \item \texttt{handle\_boundary\_violation()}: Implements Eqs.~\eqref{eq:absorbing_boundaries}--\eqref{eq:invisible_boundaries}
    \item \texttt{apply\_mutation()}: Implements Eq.~\eqref{eq:pso_mutation}
\end{itemize}

\subsubsection{GUI Integration}

The PSO implementation integrates seamlessly with the DeVana GUI through:

\begin{itemize}
    \item \textbf{Configuration Interface}: Comprehensive parameter setup in the PSO tab
    \item \textbf{Real-time Monitoring}: Progress bars, convergence plots, and status updates
    \item \textbf{Results Visualization}: Fitness evolution, parameter convergence, and performance analysis
    \item \textbf{Benchmarking Support}: Statistical analysis and comparison with other algorithms
\end{itemize}

\subsection{Performance Analysis and Benchmarking}

\subsubsection{Convergence Analysis}

The algorithm tracks multiple convergence metrics:

\begin{align}
\text{Best Fitness Evolution:} \quad & f_t^{\star} = \min_{i} f(\mathbf{x}_i(t)) \label{eq:best_fitness_evolution} \\
\text{Average Fitness Evolution:} \quad & \bar{f}_t = \frac{1}{N} \sum_{i=1}^{N} f(\mathbf{x}_i(t)) \label{eq:avg_fitness_evolution} \\
\text{Diversity Evolution:} \quad & D_t = \text{Diversity}(\{\mathbf{x}_i(t)\}_{i=1}^{N}) \label{eq:diversity_evolution}
\end{align}

\subsubsection{Computational Efficiency}

The computational complexity per iteration is:
\begin{equation}
O(N \cdot P + N \cdot C_{\text{FRF}}) \label{eq:pso_complexity}
\end{equation}

where $C_{\text{FRF}}$ is the cost of one FRF evaluation.

\subsubsection{Comparison with Other Algorithms}

PSO offers several advantages for DVA optimization:

\begin{itemize}
    \item \textbf{Continuous Optimization}: Natural fit for real-valued parameter spaces
    \item \textbf{Adaptive Behavior}: Self-tuning parameters reduce manual configuration
    \item \textbf{Diversity Maintenance}: Built-in mechanisms prevent premature convergence
    \item \textbf{Parallel Evaluation}: Particle evaluations can be parallelized
\end{itemize}

\subsection{Advanced Features Summary}

The enhanced PSO implementation in DeVana includes:

\begin{enumerate}
    \item \textbf{Adaptive Parameters}: Dynamic adjustment of inertia weight and acceleration coefficients
    \item \textbf{Multiple Topologies}: Global, Ring, Von Neumann, and Random neighborhood structures
    \item \textbf{Diversity Maintenance}: Swarm diversity monitoring and enhancement through mutation
    \item \textbf{Boundary Handling}: Multiple strategies for constraint satisfaction
    \item \textbf{Stagnation Recovery}: Detection and reinitialization of stagnant particles
    \item \textbf{Quasi-Random Initialization}: Sobol sequences for better space coverage
    \item \textbf{ML/RL Controllers}: Advanced parameter adaptation using machine learning
    \item \textbf{Comprehensive Metrics}: Detailed performance tracking and analysis
\end{enumerate}

\section{Traditional Genetic Algorithm Methodology}

    \subsection{Algorithmic Framework}
    The baseline GA follows the classic evolve–evaluate–select loop with real-valued representations tailored to the DVA design vector $\mathbf{x}$ and FRF-based fitness in Eq.~\eqref{Eq.objective_function_detailed}. Individuals are length-$P$ floating vectors, bounds are enforced by projection, and fixed parameters are respected.

    \paragraph{Representation and constraints}
    \begin{itemize}
        \item \textbf{Genome}: $\mathbf{x}=[x_1,\dots,x_P]^\top$ with $x_i\in[\theta_i^{\min},\theta_i^{\max}]$; fixed parameters are pinned to their fixed values.
        \item \textbf{Projection to bounds}: after any variation, each gene is clamped
        \begin{equation}\label{Eq.bound_projection}
            x_i \leftarrow \min\{\, \theta_i^{\max},\; \max\{\theta_i^{\min},\; x_i\}\,\},\quad i=1,\dots,P.
        \end{equation}
    \end{itemize}

    \paragraph{Fitness evaluation}
    Each individual is evaluated by the FRF pipeline (Sec.~\ref{subsubsec:chosen_performance_criteria_combined}), yielding the composite singular criterion $C_s$ in Eq.~\eqref{Eq.singular_response_detailed} and the detailed percentage differences in Eq.~\eqref{Eq.percentage_error_detailed}. The total fitness is Eq.~\eqref{Eq.objective_function_detailed}:
    \begin{equation}
        f(\mathbf{x})=\big|C_s(\mathbf{x})-1.0\big|\; +\; \alpha\sum_{k=1}^{P}|x_k|\; +\; \frac{1}{\gamma}\sum_{i,j}\big|PD_{ij}(\mathbf{x})\big|.
    \end{equation}

    \paragraph{Selection, crossover, mutation, replacement}
    \begin{itemize}
        \item \textbf{Selection}: tournament selection with size 3 (balance pressure and diversity).
        \item \textbf{Crossover}: \emph{Blend} crossover (BLX-\(\alpha\)) with $\alpha=0.5$ acts gene-wise on pairs.
        \item \textbf{Mutation}: per-gene perturbation with probability $\mathrm{indpb}=0.1$, using a uniform perturbation bounded to 10\% of each gene span; then projection (Eq.~\ref{Eq.bound_projection}).
        \item \textbf{Replacement}: generational replacement of the full population with the offspring.
    \end{itemize}

    \paragraph{Termination}
    The loop terminates when either (i) the best fitness reaches the tolerance $f_{\min}\le\texttt{ga\_tol}$, or (ii) the generation budget $G_{\max}$ is exhausted.

    \subsection{Baseline GA for DVA Design}
    \begin{algorithm}[H]
    \caption{Baseline Genetic Algorithm for DVA parameter optimization}
    \begin{algorithmic}[1]
    \REQUIRE Bounds $[\theta^{\min},\theta^{\max}]$, fixed mask/values, population size $N$, generations $G_{\max}$, $\text{cxpb}$, $\text{mutpb}$, $\text{indpb}$, tolerance $\tau$
    \STATE Initialize population $\mathcal{P}_0$ of $N$ individuals uniformly in bounds; set fixed genes.
    \FOR{$g=1$ to $G_{\max}$}
        \STATE Evaluate each $\mathbf{x}\in\mathcal{P}_{g-1}$ via FRF to compute $f(\mathbf{x})$.
        \IF{$\min_{\mathbf{x}\in\mathcal{P}_{g-1}} f(\mathbf{x})\le\tau$} \textbf{break} \ENDIF
        \STATE $\mathcal{M}\leftarrow$ tournament-select $N$ parents from $\mathcal{P}_{g-1}$ (size 3).
        \STATE Form offspring $\mathcal{O}$ by pairing parents and applying BLX-0.5 with prob. $\text{cxpb}$.
        \FOR{each offspring $\mathbf{x}\in\mathcal{O}$}
            \IF{$\text{rand}<\text{mutpb}$} mutate each gene with prob. $\text{indpb}$ by a uniform perturbation in $\pm 0.1\,(\theta_i^{\max}-\theta_i^{\min})$; enforce fixed genes and project with Eq.~\eqref{Eq.bound_projection} \ENDIF
        \ENDFOR
        \STATE Evaluate invalid individuals in $\mathcal{O}$; set $\mathcal{P}_g\leftarrow\mathcal{O}$.
    \ENDFOR
    \STATE \textbf{return} best individual and fitness.
    \end{algorithmic}
    \end{algorithm}

    \subsection{Complexity and limitations}
    Let $C_{\text{FRF}}$ be the average cost of one FRF evaluation and $N$ the population size. The time per generation is $\Theta(N\,C_{\text{FRF}})+\mathcal{O}(N)$; crossover/mutation are linear. Classical drawbacks are: (i) premature convergence under fixed operators, (ii) sensitivity to hyperparameters, (iii) costly fitness evaluations, and (iv) random seeding that may under-sample high-quality regions. These motivate the advanced controllers and screening introduced next.




\section{Advanced Genetic Algorithm Methodology}
    \subsection{Overview of Advanced Features}
    DeVana extends the baseline GA with controllers and samplers that adapt online: (i) \textbf{adaptive operator control} via an ML bandit or a lightweight RL agent that tunes crossover $\text{cxpb}$, mutation $\text{mutpb}$, and population size $N$ generation-by-generation; (ii) \textbf{advanced seeding} with Sobol/LHS low-discrepancy designs and a \textbf{NeuralSeeder} that learns promising regions using UCB or EI; and (iii) \textbf{surrogate-assisted screening} that pre-filters offspring using a KNN surrogate with novelty-based exploration. All mechanisms respect fixed parameters and bounds. Similar advanced features are also implemented for the PSO algorithm (see Sec.~\ref{sec:pso_methodology}).

    \subsection{Machine Learning Bandit Controller for Parameter Adaptation}
    The controller defines a discrete action space $\mathcal{A}=\{(\delta_{\text{cx}},\delta_{\\mu},\pi)\}$ with relative changes to $\text{cxpb}$ and $\text{mutpb}$ and a multiplier $\pi$ for population size. At generation $t$, it selects
    \begin{equation}
        a_t\;=\;\arg\max_{a\in\mathcal{A}}\Big[ \hat{R}_t(a)\; +\; c\,\sqrt{\tfrac{\ln t}{N_t(a)}} \Big], \label{Eq.ucb}
    \end{equation}
    where $\hat{R}_t(a)$ is the average reward observed for action $a$, $N_t(a)$ its count, and $c>0$ controls exploration. The chosen action updates
    \begin{align}
        \text{cxpb}_{t+1}&=\operatorname{clip}(\text{cxpb}_t\,(1+\delta_{\text{cx}}),\;[\text{cxpb}_{\min},\text{cxpb}_{\max}]),\\
        \text{mutpb}_{t+1}&=\operatorname{clip}(\text{mutpb}_t\,(1+\delta_{\mu}),\;[\text{mutpb}_{\min},\text{mutpb}_{\max}]),\\
        N_{t+1}&=\operatorname{round}(\operatorname{clip}(\pi\,N_t,\;[N_{\min},N_{\max}])).
    \end{align}
    \paragraph{Reward shaping}
    Consistent with the implementation, the reward balances improvement, speed, diversity, and effort:
    \begin{equation}\label{Eq.bandit_reward}
        R_t\;=\;\frac{\max\{0,\,f^{\star}_{t-1}-f^{\star}_{t}\}}{\max\{\epsilon,\,T_t\}}\,\frac{1}{\max\{1,\,E_t\}}\; -\; w_{\text{div}}\,\big|\mathrm{CV}_t-\mathrm{CV}^{\text{target}}\big|,
    \end{equation}
    where $f^{\star}_t$ is the best fitness in generation $t$, $T_t$ is generation time, $E_t$ the number of evaluations, and $\mathrm{CV}_t=\tfrac{\sigma_t}{|\mu_t|+\epsilon}$ is the coefficient of variation of fitness. A blended estimator $\tilde R_t= w_{\text{hist}}\,\hat{R}_{t-1}(a)+w_{\text{cur}}\,R_t$ stabilizes updates.

    \begin{algorithm}[H]
    \caption{ML-Bandit controller (per generation)}
    \begin{algorithmic}[1]
    \STATE Compute statistics $\mu_t,\sigma_t,f^{\star}_t,T_t,E_t$ from the finished generation.
    \STATE For each $a\in\mathcal{A}$, compute UCB score via Eq.~\eqref{Eq.ucb}; select $a_t$.
    \STATE Update $(\text{cxpb},\text{mutpb},N)$ by the selected action with clipping.
    \STATE Observe $R_t$ via Eq.~\eqref{Eq.bandit_reward}, update counts and averages for $a_t$.
    \end{algorithmic}
    \end{algorithm}

    \subsection{Reinforcement Learning-Based Control of GA Parameters}
    A compact Q-learning agent with an $\epsilon$-greedy policy selects actions $a\in\mathcal{A}$ conditioned on a coarse state $s$ (e.g., improvement/no-improvement). The update is
    \begin{equation}
        Q(s,a)\leftarrow Q(s,a)+\alpha\,\big(R+\gamma\max_{a'}Q(s',a')-Q(s,a)\big),
    \end{equation}
    with learning rate $\alpha$, discount $\gamma$, and exploration $\epsilon$ decayed each generation. Actions map to the same $(\delta_{\text{cx}},\delta_{\mu},\pi)$ triplets and are clipped to guardrails as above.

    \begin{algorithm}[H]
    \caption{RL controller (per generation)}
    \begin{algorithmic}[1]
    \STATE Observe state $s_t$ (e.g., $\mathbb{1}[f^{\star}_t<f^{\star}_{t-1}]$).
    \STATE With prob. $\epsilon$ select random $a_t$, else $\arg\max_a Q(s_t,a)$.
    \STATE Apply $a_t$ to update $(\text{cxpb},\text{mutpb},N)$ with clipping.
    \STATE Receive reward $R_t$ per Eq.~\eqref{Eq.bandit_reward}, observe $s_{t+1}$; update $Q$.
    \STATE Decay $\epsilon\leftarrow\epsilon\cdot\epsilon_{\text{decay}}$.
    \end{algorithmic}
    \end{algorithm}

    \subsection{Sobol and Latin Hypercube Seeding}
    For $d=P$ free parameters with bounds $\ell_i,u_i$, a QMC engine generates $m$ points $\mathbf{z}_k\in[0,1]^d$ using either Sobol or LHS, then scales
    \begin{equation}
        x_{k,i}\;=\;\ell_i + z_{k,i}\,(u_i-\ell_i),\quad i=1,\dots,d,\;k=1,\dots,m,
    \end{equation}
    and overwrites fixed coordinates. QMC seeding improves space-filling vs. pure random initialization.

    \subsection{Neural Network-Based Population Seeding (NeuralSeeder)}
    The NeuralSeeder maintains a dataset $\mathcal{D}=\{(\mathbf{x}^{(n)}, y^{(n)})\}$ of evaluated points and trains a lightweight ensemble to model $y\approx f(\mathbf{x})$. It proposes candidates by optimizing an acquisition function such as UCB or EI over a candidate pool, with exploration fraction $\epsilon$ and UCB scale $\beta$ adapted online:
    \begin{align}\label{eq:UCBEI}
        \text{UCB}(\mathbf{x}) &= \mu(\mathbf{x}) - \beta\,\sigma(\mathbf{x}) \quad \text{(minimization)},\\
        \text{EI}(\mathbf{x}) &= \mathbb{E}\big[\max\{0, f^{\star}-Y(\mathbf{x})\}\big].
    \end{align}
    A diversity constraint (minimum distance in normalized space) avoids mode collapse. When the controller resizes the population, additional individuals are drawn from NeuralSeeder once it has seen enough data; otherwise it falls back to the configured seeding method.

    \subsection{Surrogate Model-Assisted Fitness Evaluation}
    To reduce FRF calls, a KNN surrogate pre-screens offspring. Let $\tilde{\mathbf{x}}$ be a candidate and define normalized coordinates $z_i=(x_i-\ell_i)/(u_i-\ell_i)$. The KNN predictor is
    \begin{equation}
        \hat f(\tilde{\mathbf{x}})= \frac{1}{k}\sum_{(\mathbf{x}^{(n)},y^{(n)})\in\mathcal{N}_k(\tilde{\mathbf{x}})} y^{(n)},\quad \mathcal{N}_k(\tilde{\mathbf{x}})=\arg\min_{\mathcal{S},\;|\mathcal{S}|=k}\sum_{(\mathbf{x},\cdot)\in\mathcal{S}}\lVert \mathbf{z}-\tilde{\mathbf{z}}\rVert_2.
    \end{equation}
    A pool of size $M=\lceil \phi\,Q\rceil$ (with $Q$ invalid offspring and pool factor $\phi\ge1$) is generated by cloning and light genetic operations; the top $q$ by $\hat f$ are \emph{exploited} and an \emph{explore} subset is added by maximizing novelty
    \begin{equation}
        \mathcal{N}(\tilde{\mathbf{x}})=\min_{\mathbf{x}^{(n)}\in\mathcal{D}}\lVert \tilde{\mathbf{z}}-\mathbf{z}^{(n)}\rVert_2.
    \end{equation}
    Only the chosen subset is evaluated by FRF; others are discarded, yielding substantial savings when FRF is expensive.

    \begin{algorithm}[H]
    \caption{Surrogate screening per generation}
    \begin{algorithmic}[1]
    \REQUIRE Invalid offspring set $\mathcal{U}$ of size $Q$, pool factor $\phi$, explore fraction $\eta$, KNN $k$
    \STATE Build pool $\mathcal{P}$ of size $M=\max\{Q,\lceil\phi Q\rceil\}$ by cloning $\mathcal{U}$ and applying light crossover/mutation (with bounds and fixed genes).
    \STATE Predict $\hat f$ for all $\mathbf{x}\in\mathcal{P}$; sort ascending.
    \STATE Select $q=Q$ exploit candidates with smallest $\hat f$.
    \STATE From the remainder, select $\eta q$ most novel by maximizing $\mathcal{N}(\cdot)$; add to chosen set.
    \STATE Evaluate FRF only on the chosen set; fill remaining offspring slots by best evaluated if needed.
    \end{algorithmic}
    \end{algorithm}

    \subsection{Legacy adaptive rates}
    As a fallback, a heuristic adapts $(\text{cxpb},\text{mutpb})$ based on diversity $\sigma/|\mu|$ and stagnation counter: low diversity increases mutation and decreases crossover, high diversity does the opposite, with periodic adjustments.


\section{Benchmarking and Robustness Analysis}
\label{sec:benchmarking}
DeVana instruments the optimization with detailed metrics to quantify efficiency and robustness across controllers and seeding strategies.

\subsection{Recorded metrics}
\begin{itemize}
    \item \textbf{Per generation}: best/mean/std fitness, population size, $(\text{cxpb},\text{mutpb})$, evaluations, time breakdown (selection/crossover/mutation/evaluation), total generation time.
    \item \textbf{System}: CPU\%, per-core CPU, memory usage/details, I/O counters, network, thread count.
    \item \textbf{Controllers}: action histories and rewards for ML-bandit/RL; rate/adaptation history.
\end{itemize}

\subsection{Key performance indicators (KPIs)}
Let $f^{\star}_g$ be best fitness at generation $g$ and $G$ the total generations.
\begin{align}
    g_{\tau} &= \min\{g:\ f^{\star}_g\le\tau\}\quad\text{(time-to-tolerance)},\\
    \mathrm{AUC} &= \sum_{g=1}^{G} f^{\star}_g\quad\text{(area under best-fitness curve; lower is better)},\\
    \bar T &= \frac{1}{G}\sum_{g=1}^{G} T_g,\quad \bar E = \frac{1}{G}\sum_{g=1}^{G} E_g,\\
    \Delta f_g &= \max\{0, f^{\star}_{g-1}-f^{\star}_g\},\quad \overline{\Delta f}=\frac{1}{G-1}\sum_{g=2}^{G}\Delta f_g.
\end{align}
Reliability is reported as the fraction of runs achieving $f^{\star}_G\le\tau$.

\subsection{Benchmark design}
We adopt a factorial study over controllers (Fixed/Adaptive/ML-Bandit/\gls{rl}) and seeders (Random/Sobol/LHS/Neural), with and without surrogate screening. For each cell, run $R$ replicates with different seeds; collect KPIs and resource metrics.

\paragraph{Ablation studies}
We perform ablations by enabling one advanced feature at a time: (i) adaptive controllers only; (ii) surrogate screening only; (iii) advanced seeding only; and (iv) combinations. This isolates the marginal impact of each component.

\paragraph{Sensitivity analyses}
We probe sensitivity to: (a) controller exploration constants, (b) surrogate pool factor $\phi$ and $k$, (c) NeuralSeeder exploration fraction and acquisition scale, and (d) objective weights. Outcomes are reported on ${g_\tau}$, AUC, and final best fitness.

\paragraph{External validity}
We validate across multiple mechanical scenarios (mass ratios, damping regimes, and different target criteria) to ensure conclusions generalize beyond a single configuration.

\subsection{Statistical analysis}
Report medians and IQRs across replicates; compare cells using non-parametric tests (e.g., Wilcoxon) and bootstrap 95\% CIs for median differences of AUC and final best fitness. When multiple factors are varied, use aligned ranks or a permutation-based two-way ANOVA analogue on AUC.

\subsection{Reporting}
Provide: (i) convergence traces with shaded IQR, (ii) boxplots of KPIs, (iii) Pareto plots of final fitness vs. wall-time/evaluations, (iv) controller rate/population trajectories, (v) resource timelines, and (vi) ablation/sensitivity summaries.

\section{Case studies and validation}
\label{sec:case-studies}
We demonstrate \softwareName{} on representative scenarios: (i) baseline 2DOF--3DOF configuration with two primary targets; (ii) high-damping regime; (iii) multi-peak realignment with stringent slope uniformity; and (iv) sensitivity to inerter couplings. For each case we compare the algorithm suite, with and without advanced features, and report KPIs and FRF visualizations.

\subsection{Experimental setup}
We fix bounds, targets, and weight schema per Sec.~\ref{subsubsec:chosen_performance_criteria_combined}. Each optimizer uses identical stopping criteria and evaluation budgets. Controllers and surrogates follow their default guardrails unless stated.

\subsection{Results overview}
Across cases, adaptive \gls{ga} controllers reduce time-to-tolerance and evaluations, surrogate screening yields the largest savings when FRF is most expensive, and NeuralSeeder improves early convergence and reliability. Results are detailed in Sec.~\ref{sec:benchmarking}.

\section{Statistical synthesis of recommended DVA parameter ranges}
\label{sec:statistical-ranges}

\subsection{Motivation and overview}
Multiple independent GA runs provide a sample of optimized parameter vectors that encode how the system converges under different random seeds, seeding strategies, and adaptive controllers. Rather than reporting a single point estimate, we construct statistically robust \emph{recommended ranges} for every design parameter. These ranges capture where high-performing solutions concentrate, improve interpretability, and guide robust engineering choices.

This section details a step-by-step, statistically grounded procedure that aggregates results from many runs and produces parameter ranges using several complementary criteria. It also defines how to compare, combine, and visualize these ranges, and it gives principled rules for selecting the final recommended interval per parameter.

\subsection{Data model and notation}
Let there be $R$ benchmark runs of the GA. Each run $r\in\{1,\dots,R\}$ returns:
\begin{itemize}
    \item Best solution vector $\mathbf{x}^{(r)} = [x^{(r)}_1,\dots, x^{(r)}_P]^\top$ of dimension $P$ (number of design parameters),
    \item Best fitness value $f^{(r)}$ (lower is better),
    \item Parameter name list $\{p_1,\dots,p_P\}$ consistent across runs.
\end{itemize}
For each parameter $p_j$ we collect the empirical sample
\begin{equation}
    \mathcal{X}_{p_j} = \{\, x^{(1)}_j, x^{(2)}_j, \dots, x^{(R)}_j \,\},\qquad j=1,\dots,P.
\end{equation}
We also define an optional performance subset using the $q$-quantile of fitness:
\begin{equation}
    \mathcal{I}_{q} = \{\, r:\ f^{(r)} \leq Q_f(q) \,\},\quad \mathcal{X}^{q}_{p_j}=\{\, x^{(r)}_j:\ r\in \mathcal{I}_{q}\,\},
\end{equation}
where $Q_f(q)$ is the $q$-quantile of $\{f^{(r)}\}_{r=1}^R$.

\subsection{Preprocessing and quality control}
Prior to range estimation we apply the following checks:
\begin{enumerate}[label=\textbf{P\arabic*}]
    \item \textbf{Validation}: discard runs with missing or incompatible dimension $P$, or non-finite entries.
    \item \textbf{Alignment}: ensure parameter names $\{p_j\}$ align across runs.
    \item \textbf{Optional winsorization}: for visual diagnostics only, we may clip the top/bottom 1\% to prevent extreme values from dominating plots; the statistical criteria below are robust and do not require it.
\end{enumerate}

\subsection{Range estimation criteria}
To provide statistically meaningful and robust recommended ranges for each design parameter $p_j$, we compute six complementary interval criteria. Each criterion yields an interval $[L^{(c)}_j,\,U^{(c)}_j]$ for parameter $p_j$, where $c$ indexes the criterion. These criteria are implemented in the GUI (see \texttt{codes/gui/main\_window/ga\_mixin.py}, e.g., \texttt{\_iqr\_range}, \texttt{\_p5\_p95}, \texttt{\_tukey\_whisker}, \texttt{\_shortest\_interval}, \texttt{\_top\_quantile\_p5\_p95}, \texttt{\_trimmed\_mean\_mad}) and operate directly on the empirical samples of each parameter, as produced by \texttt{GAWorker.finished}. Below, we describe each criterion in detail, including the meaning of all parameters and their statistical motivation.

\subsubsection{Interquartile Range (IQR, Q1--Q3)}
\vspace{-0.25em}
\begin{equation}
    L^{\text{IQR}}_j = Q_{x_j}(0.25), \qquad U^{\text{IQR}}_j = Q_{x_j}(0.75).
\end{equation}
Here, $Q_{x_j}(q)$ denotes the $q$-th quantile of the sample $\mathcal{X}_{p_j}$ for parameter $p_j$. The IQR interval captures the central 50\% of the observed values for $p_j$, i.e., the range between the first quartile (25th percentile) and the third quartile (75th percentile). This method is highly robust to outliers and is especially effective when the distribution of $p_j$ is unimodal or only mildly skewed. By focusing on the middle half of the data, it provides a compact summary of where most solutions concentrate, but it intentionally ignores the tails of the distribution.

\subsubsection{P5--P95 Quantile Interval}
\vspace{-0.25em}
\begin{equation}
    L^{\text{P5--P95}}_j = Q_{x_j}(0.05), \qquad U^{\text{P5--P95}}_j = Q_{x_j}(0.95),
\end{equation}
This interval is defined by the 5th and 95th percentiles of the sample for parameter $p_j$. It includes the central 90\% of the data, thus providing a broader coverage than the IQR. The P5--P95 interval is moderately robust to outliers: it trims the most extreme 5\% of values at each end, which is useful for distributions with moderately heavy tails. This criterion is often preferred when stakeholders desire a more generous envelope that still excludes the most extreme solutions.

\subsubsection{Tukey Whiskers (Classical Outlier Trimming)}
\begin{align}
    \left[ L^{\text{Tukey}}_j,\, U^{\text{Tukey}}_j \right] &= \left[
        \min\left\{ x \in \mathcal{X}_{p_j} : x \ge Q_1 - 1.5\,\mathrm{IQR} \right\},\;
        \max\left\{ x \in \mathcal{X}_{p_j} : x \le Q_3 + 1.5\,\mathrm{IQR} \right\}
    \right]
\end{align}
This approach trims classical outliers—values that fall outside the Tukey fences—while preserving the full variability of the interior data. It is especially interpretable and useful when a few extreme runs would otherwise distort the recommended range, but you still want to include as much of the “typical” data as possible.

\subsubsection{Shortest $\alpha$ High-Density Interval (HDI)}
The HDI criterion seeks the most compact interval containing a specified fraction $\alpha$ of the data. For parameter $p_j$, let $\alpha\in(0,1)$ (we use $\alpha=0.68$ by default, analogous to a 1-sigma interval for a normal distribution). Sort the sample values $v_1\le\dots\le v_n$ and set $k=\big\lfloor \alpha n\big\rceil$. The HDI is the shortest interval among all contiguous windows of length $k$:
\begin{equation}
    [L^{\text{HDI}}_j, U^{\text{HDI}}_j] = \operatorname*{arg\,min}_{1\le s\le n-k+1} (v_{s+k-1}-v_s).
\end{equation}
This method is non-parametric and adapts to the actual shape of the sample distribution, making it especially effective for skewed or multi-modal data. The HDI focuses on the densest region of solutions, which is often where the most promising or robust parameter values lie.

\subsubsection{Top-$q$ Performance P5--P95}
This criterion restricts attention to the best-performing fraction of runs, as measured by the fitness value $f^{(r)}$ for each run $r$. Let $q\in(0,1)$ denote the top fraction to consider (default $q=0.25$, i.e., the best 25\% of runs). Define the index set $\mathcal{I}_q = \{ r:\ f^{(r)} \leq Q_f(q) \}$, where $Q_f(q)$ is the $q$-th quantile of the fitness values. The subset of parameter samples from these top runs is $\mathcal{X}^{q}_{p_j}$. The interval is then:
\begin{equation}
    L^{\text{Top}}_j = Q_{x_j\in\mathcal{X}^{q}_{p_j}}(0.05),\qquad U^{\text{Top}}_j = Q_{x_j\in\mathcal{X}^{q}_{p_j}}(0.95).
\end{equation}
This approach highlights the range where high-performing solutions for $p_j$ are concentrated, making it particularly useful when the goal is to recommend parameter values that are most likely to yield elite performance. Note that this criterion is sensitive to the choice of $q$ and requires a sufficient number of top-performing samples for stability.

\subsubsection{Trimmed Mean $\pm$ 1.5\,MAD (TMAD, Robust)}
This robust criterion combines the trimmed mean and the median absolute deviation (MAD) to define a stable interval, even for small sample sizes or heavy-tailed distributions. For parameter $p_j$, let $m=\mathrm{median}(\mathcal{X}_{p_j})$ and $\mathrm{MAD}=\mathrm{median}(|x-m|)$, which measures the typical deviation from the median. The robust scale estimate is $\sigma_{\text{rob}}=1.4826\,\mathrm{MAD}$ (the constant makes it consistent with the standard deviation for normal data). Optionally, a 10\% symmetric trimming is applied before computing the mean $\tilde\mu$ (i.e., the mean of the central 80\% of values). The interval is:
\begin{equation}
    L^{\text{TMAD}}_j = \max\{\min\mathcal{X}_{p_j},\; \tilde\mu - 1.5\,\sigma_{\text{rob}}\},\quad
    U^{\text{TMAD}}_j = \min\{\max\mathcal{X}_{p_j},\; \tilde\mu + 1.5\,\sigma_{\text{rob}}\}.
\end{equation}
This method yields compact, outlier-resistant intervals and is especially reliable when the number of runs $R$ is small or the data are contaminated by outliers.

\subsection{Data ingestion and aggregation (GAWorker $\rightarrow$ GUI)}
\label{subsec:ingestion}
Each optimization run is executed by \texttt{GAWorker.run()}, which, upon completion, emits a set of outputs via \texttt{GAWorker.finished}:
\begin{itemize}
    \item \texttt{final\_results}: The full results of the run, optionally including \texttt{singular\_response} (detailed response data) and \texttt{benchmark\_metrics} (summary statistics and metadata).
    \item \texttt{best\_ind}: The best solution vector (i.e., the parameter values for the best individual found in the run).
    \item \texttt{parameter\_names}: An ordered list of parameter names, ensuring consistent mapping across runs.
    \item \texttt{best\_fitness}: The fitness value associated with the best solution.
\end{itemize}
The GUI mixin aggregates these outputs into a structured table, where each row corresponds to a run and columns include \texttt{run\_id}, \texttt{fitness}, and one column for each parameter (named according to \texttt{parameter\_names}). Additional metadata—such as the seeding method (\texttt{seeding\_method}), controller type (e.g., fixed, adaptive, ML-bandit, RL), and surrogate model settings—are attached from \texttt{benchmark\_metrics} to enable filtering and stratified analysis. Parameters that are fixed (i.e., have collapsed bounds and do not vary across runs) are automatically detected and treated as degenerate intervals (single values).

Special edge cases handled by the GUI include:
\begin{itemize}
    \item \textbf{Fixed parameters:} If a parameter is fixed across all runs, all criteria return the degenerate interval $[\theta^{\min},\theta^{\max}]$ for that parameter, reflecting its lack of variability.
    \item \textbf{Insufficient top-performing samples:} If the Top-$q$ subset contains fewer than 5 samples, the Top-$q$ P5--P95 interval is considered unstable and is omitted in favor of more robust criteria (HDI, IQR, TMAD).
\end{itemize}

\subsection{Criterion robustness and guidance}
Each interval criterion has distinct statistical properties and is suited to different data characteristics:
\begin{itemize}
    \item \textbf{IQR:} Highly robust to outliers and provides a compact interval. It is a good default when the parameter distribution is unimodal or only mildly skewed. By design, it ignores the tails, focusing on the central bulk of solutions.
    \item \textbf{P5--P95:} Offers broader coverage than IQR, including 90\% of the data. It is moderately robust and is appropriate when a more generous envelope is desired, while still trimming the most extreme values.
    \item \textbf{Tukey:} Implements classical outlier trimming using the 1.5 IQR rule. It is interpretable and effective when a few extreme runs would otherwise distort the range, but you want to retain as much of the “typical” data as possible.
    \item \textbf{HDI$_{0.68}$:} Finds the shortest interval containing 68\% of the data, regardless of the distribution’s shape. It excels for skewed or multi-modal samples and is especially useful for identifying dense “sweet spots” in the parameter space.
    \item \textbf{Top-$q$ P5--P95:} Focuses on the best-performing fraction of runs (e.g., top 25\% by fitness). This criterion is ideal when the recommended range should reflect where elite solutions are concentrated, but it is sensitive to the choice of $q$ and requires a sufficient number of top-performing samples.
    \item \textbf{TMAD:} Based on the median and MAD, with optional trimming, this is the most stable criterion for small sample sizes or when the data are heavy-tailed. It is conservative but highly reliable, making it a strong default in challenging scenarios.
\end{itemize}
\textbf{Recommended usage:} For most applications, begin with TMAD or IQR for stability and robustness. Add HDI to capture the densest region of solutions, especially if the distribution is skewed or multi-modal. Report P5--P95 for a broad, stakeholder-friendly envelope. Use Top-$q$ to highlight elite performance corridors, and apply Tukey when explicit, interpretable outlier handling is required.

\subsection{Algorithms}
Below we provide efficient, implementation-ready pseudocode for two of the more complex criteria, matching the helper functions in \texttt{ga\_mixin.py}. All variables are defined for clarity.

\begin{algorithm}[H]
\caption{Shortest $\alpha$-HDI for a parameter sample}
\begin{algorithmic}[1]
\REQUIRE Sample $\mathcal{X}=\{x_1,\dots,x_n\}$ for parameter $p_j$; desired coverage $\alpha\in(0,1)$ (default $0.68$)
\STATE $v \leftarrow \mathrm{sort}(\mathcal{X})$ ascending \hfill // Sort the sample values
\STATE $k \leftarrow \max\{1, \lfloor \alpha n \rceil\}$ \hfill // Number of points in the interval
\STATE $w^\star \leftarrow +\infty$, $s^\star \leftarrow 1$ \hfill // Initialize best width and start index
\FOR{$s=1$ to $n-k+1$}
    \STATE $w \leftarrow v_{s+k-1} - v_s$ \hfill // Width of current window
    \IF{$w < w^\star$} $w^\star\leftarrow w$, $s^\star\leftarrow s$ \ENDIF
\ENDFOR
\STATE \textbf{return} $[v_{s^\star},\; v_{s^\star+k-1}]$ \hfill // Shortest interval found
\end{algorithmic}
\end{algorithm}

\begin{algorithm}[H]
\caption{Top-$q$ performance P5--P95}
\begin{algorithmic}[1]
\REQUIRE Paired samples $\{(x^{(r)}, f^{(r)})\}_{r=1}^R$ for parameter $p_j$ and fitness; top fraction $q\in(0,1)$ (e.g., $q=0.25$)
\STATE $\tau \leftarrow Q_{\{f^{(r)}\}}(q)$ \hfill // Fitness threshold for top $q$ fraction
\STATE $\mathcal{J} \leftarrow \{ r:\ f^{(r)} \le \tau \}$ \hfill // Indices of top-performing runs
\STATE $\mathcal{X}^{q} \leftarrow \{ x^{(r)}:\ r\in\mathcal{J}\}$ \hfill // Parameter values from top runs
\STATE $L \leftarrow Q_{\mathcal{X}^{q}}(0.05)$, $U \leftarrow Q_{\mathcal{X}^{q}}(0.95)$ \hfill // 5th and 95th percentiles
\STATE \textbf{return} $[L, U]$
\end{algorithmic}
\end{algorithm}

\subsection{Aggregation, comparison, and decision rules}
Once all per-criterion intervals $[L^{(c)}_j, U^{(c)}_j]$ are computed for each parameter $p_j$, we aggregate and compare them using the following metrics and rules:
\begin{itemize}
    \item \textbf{Width and center:} For each criterion $c$, the width is $W^{(c)}_j=U^{(c)}_j-L^{(c)}_j$ and the center is $C^{(c)}_j=(U^{(c)}_j+L^{(c)}_j)/2$. These summarize the size and location of each interval.
    \item \textbf{Union band:} The union interval $[L^{\cup}_j, U^{\cup}_j]=[\min_c L^{(c)}_j,\; \max_c U^{(c)}_j]$ covers all values included by any criterion.
    \item \textbf{Intersection band:} The intersection $[L^{\cap}_j, U^{\cap}_j]=[\max_c L^{(c)}_j,\; \min_c U^{(c)}_j]$ is the region where all intervals overlap (if non-empty).
    \item \textbf{Majority intersection:} To avoid bias from a single criterion, we can intersect only those intervals supported by at least $M$ criteria (e.g., $M=3$ out of 6).
    \item \textbf{Consensus score:} $S^{\text{cons}}_j = \dfrac{\max\{0,\,U^{\cap}_j-L^{\cap}_j\}}{\max\{\varepsilon,\,U^{\cup}_j-L^{\cup}_j\}}$, where $\varepsilon\ll1$ prevents division by zero. This score quantifies the degree of agreement among criteria (1 = perfect consensus, 0 = no overlap).
    \item \textbf{Pairwise Intersection-over-Union (IoU):} For any two criteria $c_1$ and $c_2$, $\mathrm{IoU}^{(c_1,c_2)}_j = \dfrac{\max\{0,\,\min(U^{(c_1)}_j,U^{(c_2)}_j)-\max(L^{(c_1)}_j,L^{(c_2)}_j)\}}{\max\{\varepsilon,\,\max(U^{(c_1)}_j,U^{(c_2)}_j)-\min(L^{(c_1)}_j,L^{(c_2)}_j)\}}$ measures the overlap between intervals.
    \item \textbf{Normalized width:} $\tilde W^{(c)}_j = W^{(c)}_j/\max\{\varepsilon,\,U^{\cup}_j-L^{\cup}_j\}$ expresses each interval’s width as a fraction of the union width, allowing direct comparison of compactness across criteria.
\end{itemize}

\noindent\textbf{Recommended range selection.} The following decision rules are used to select the final recommended interval for each parameter $p_j$:
\begin{enumerate}[label=\textbf{R\arabic*}]
    \item If the intersection band $[L^{\cap}_j, U^{\cap}_j]$ is non-empty and its width $W^{\cap}_j$ exceeds a minimal engineering tolerance, use this as the recommended range.
    \item If the intersection is empty or too narrow, use the \emph{majority intersection} (intersection of at least $M$ criteria). If this is also empty, fall back to the shortest width $W^{(c)}_j$ among the robust criteria (HDI, IQR, TMAD).
    \item For parameters that show strong sensitivity to performance (e.g., high correlation with fitness), prefer the \emph{Top-$q$} or \emph{HDI} interval, as these focus on elite or dense regions.
    \item Always enforce physical or engineering bounds: clamp the final recommended interval $[L_j, U_j]$ to the allowed range $[\theta_j^{\min}, \theta_j^{\max}]$ for parameter $p_j$.
\end{enumerate}

\subsection{Diagnostics and visualization}
To support interpretation and transparency, we generate two main types of diagnostics for each parameter:
\begin{enumerate}
    \item \textbf{Stacked interval bands:} For each parameter, we plot stacked, color-coded rectangles representing the intervals from all criteria. This visualization makes it easy to see where criteria agree or disagree, and to compare the widths and locations of the intervals.
    \item \textbf{Width heatmap:} We construct a matrix where each row is a parameter and each column is a criterion, with entries showing the interval width $W^{(c)}_j$ (optionally normalized by the union width). This heatmap highlights which parameters are tightly constrained (stable) and which are more uncertain.
\end{enumerate}

\subsection{Statistical guidance on criterion choice}
The choice of interval criterion should be guided by the characteristics of the data and the goals of the analysis:
\begin{itemize}
    \item \textbf{Small sample size ($R$) or heavy outliers:} Prefer \emph{TMAD} and \emph{IQR}, as these are most robust and stable.
    \item \textbf{Skewed or multi-modal parameter distributions:} Use \emph{HDI}, which adapts to the actual density and can capture multiple modes or asymmetry.
    \item \textbf{When the recommended range should reflect high performance:} Use \emph{Top-$q$}, which focuses on the best-performing solutions.
    \item \textbf{When broader coverage is acceptable or desired:} Use \emph{P5--P95}, which includes most of the data while trimming only the extremes.
    \item \textbf{When explicit, interpretable outlier handling is needed:} Use \emph{Tukey}, which applies a classical, well-understood rule for outlier exclusion.
\end{itemize}

\subsection{Complexity and implementation notes}
The computational complexity of the statistical range synthesis procedure is dominated by the sorting operations required for quantile-based criteria. For each parameter, most criteria (such as IQR, P5--P95, Tukey, and HDI) require sorting the $R$ values collected from independent optimization runs, resulting in a per-parameter complexity of $O(R \log R)$. Simple quantile calculations (e.g., computing the median or a single percentile) can be performed in $O(R)$ time using selection algorithms, but in practice, sorting is used for flexibility and simplicity, especially when multiple quantiles or more complex intervals (like HDI or Tukey) are needed.

Since the procedure is repeated for each of the $P$ parameters, the total computational cost scales as $O(P R \log R)$. This linear scaling in the number of parameters $P$ and near-linear scaling in the number of runs $R$ makes the method practical and efficient even for high-dimensional problems or when aggregating results from hundreds of optimization runs. 

In the software implementation, the GUI leverages efficient, vectorized operations provided by NumPy and Pandas. These libraries allow for rapid computation of quantiles, sorting, and interval construction across all parameters and runs. The mapping from the mathematical definitions of each criterion to code is direct: for example, the IQR is computed using \texttt{numpy.percentile} or \texttt{pandas.Series.quantile}, and the HDI is implemented via a sliding window over the sorted samples. This ensures that the statistical analysis remains both robust and performant, supporting interactive exploration and visualization in the GUI.

\subsection{End-to-end procedure (from optimization to ranges and comparison)}
The following algorithm outlines the complete workflow, starting from the outputs of repeated optimization runs and culminating in the generation, comparison, and export of recommended parameter ranges. Each step is designed to ensure traceability, robustness, and transparency in the statistical synthesis process.

\begin{algorithm}[H]
\caption{Aggregate and Validate Optimization Run Data}
\begin{algorithmic}[1]
\REQUIRE $R$ independent optimization runs executed via \texttt{GAWorker.run()}; each run outputs a tuple \texttt{(final\_results, best\_ind, parameter\_names, best\_fitness)}; parameter bounds $[\theta_j^{\min},\theta_j^{\max}]$ for all $j$; optional metadata (e.g., seeding method, controller type, surrogate usage)
\STATE \textbf{Data collection:} In the GUI, aggregate all runs into a structured table. Each row corresponds to a run and contains the run identifier (\texttt{run\_id}), the final fitness value (\texttt{fitness}), and one column for each parameter in \texttt{parameter\_names[j]}.
\STATE \textbf{Validation and alignment:} Ensure that parameter names and ordering are consistent across all runs. Remove any runs that are incomplete, invalid, or contain missing data to maintain data integrity.
\STATE \textbf{Sample construction:} For each parameter $p_j$, extract the set of observed values across all runs, forming the sample $\mathcal{X}_{p_j} = \{x^{(r)}_j\}_{r=1}^R$. Simultaneously, collect the corresponding fitness values $\{f^{(r)}\}_{r=1}^R$. Identify parameters that are fixed (i.e., have the same value in all runs).
\STATE \textbf{Top-$q$ subset (optional):} If desired, define a subset of runs corresponding to the top-performing fraction $q$ (e.g., $q=0.25$). Compute the fitness threshold $\tau = Q_f(q)$, and let $\mathcal{I}_q = \{ r : f^{(r)} \leq \tau \}$ be the indices of the top runs. For each parameter, extract the corresponding values $\mathcal{X}^{q}_{p_j}$.
\end{algorithmic}
\end{algorithm}

\begin{algorithm}[H]
\caption{Compute Statistical Intervals for Each Parameter}
\begin{algorithmic}[1]
\REQUIRE For each parameter $p_j$, sample $\mathcal{X}_{p_j}$ (and optionally $\mathcal{X}^{q}_{p_j}$), parameter bounds $[\theta_j^{\min}, \theta_j^{\max}]$
\FOR{$j=1$ to $P$}
    \IF{$p_j$ is fixed}
        \STATE For all criteria, set the interval to the full allowed range $[\theta_j^{\min}, \theta_j^{\max}]$ (since no variability is observed), and \textbf{continue} to the next parameter.
    \ENDIF
    \STATE \textbf{Interval computation:} For parameter $p_j$, compute the recommended intervals according to each statistical criterion using GUI helper functions:
        \begin{itemize}
            \item \textbf{IQR:} Interquartile range (25th to 75th percentile)
            \item \textbf{P5--P95:} 5th to 95th percentile interval
            \item \textbf{Tukey:} Tukey's outlier rule (using $1.5\times$ IQR)
            \item \textbf{HDI$_{0.68}$:} Highest Density Interval containing 68\% of the data
            \item \textbf{Top-$q$ P5--P95:} 5th to 95th percentile within the top-$q$ subset (if $|\mathcal{I}_q|$ is sufficiently large)
            \item \textbf{TMAD:} Trimmed Median Absolute Deviation interval
        \end{itemize}
\ENDFOR
\end{algorithmic}
\end{algorithm}

\begin{algorithm}[H]
\caption{Compute Interval Metrics and Compare Criteria}
\begin{algorithmic}[1]
\REQUIRE For each parameter $p_j$, intervals $[L^{(c)}_j, U^{(c)}_j]$ for all criteria $c$
\FOR{$j=1$ to $P$}
    \STATE For each criterion $c$, calculate the interval width $W^{(c)}_j = U^{(c)}_j - L^{(c)}_j$ and center $C^{(c)}_j = (U^{(c)}_j + L^{(c)}_j)/2$.
    \STATE Compute the union interval $[L^{\cup}_j, U^{\cup}_j]$ (spanning all criteria) and the intersection interval $[L^{\cap}_j, U^{\cap}_j]$ (where all intervals overlap).
    \STATE Quantify agreement and differences among criteria by computing:
        \begin{itemize}
            \item \textbf{Consensus score} $S^{\text{cons}}_j$ (fraction of overlap among all intervals)
            \item \textbf{Pairwise Intersection-over-Union (IoU)} between all pairs of criteria
            \item \textbf{Normalized widths} $\tilde W^{(c)}_j$ (width of each interval relative to the union width)
        \end{itemize}
\ENDFOR
\end{algorithmic}
\end{algorithm}

\begin{algorithm}[H]
\caption{Select Recommended Parameter Ranges}
\begin{algorithmic}[1]
\REQUIRE For each parameter $p_j$, all computed intervals, metrics, and parameter bounds $[\theta_j^{\min}, \theta_j^{\max}]$
\FOR{$j=1$ to $P$}
    \STATE Apply the decision rules (R1--R4) to select the final recommended interval $[L_j, U_j]$ for parameter $p_j$. This may involve using the intersection, majority intersection, or the most robust criterion, and always clamping to the physical bounds $[\theta_j^{\min}, \theta_j^{\max}]$.
\ENDFOR
\end{algorithmic}
\end{algorithm}

\begin{algorithm}[H]
\caption{Visualize, Report, and Export Results}
\begin{algorithmic}[1]
\REQUIRE For all parameters, per-criterion intervals, metrics, and final recommended ranges
\STATE Present the results in the GUI as follows:
    \begin{itemize}
        \item Per-criterion tables showing $[L^{(c)}_j, U^{(c)}_j, W^{(c)}_j, C^{(c)}_j]$ for all parameters
        \item Stacked-interval plots visualizing the intervals for each parameter and criterion
        \item Overlays comparing intervals across criteria, width heatmaps, and summary metrics
    \end{itemize}
\STATE Export all computed intervals, union/intersection bands, comparison metrics, and final recommended ranges $[L_j, U_j]$ to CSV files. These are saved alongside the \texttt{benchmark\_metrics} for full traceability and reproducibility of the analysis.
\end{algorithmic}
\end{algorithm}

\subsection{Mapping to software implementation}
The GUI adds a \emph{Parameter Ranges} benchmarking tab with one subtab per criterion and a \emph{Compare Criteria} subtab. Each subtab presents: (i) a table of $[L^{(c)}_j, U^{(c)}_j, W^{(c)}_j, C^{(c)}_j]$ across all parameters; (ii) a horizontal stacked-rectangle visualization; and (iii) export options (CSV). The comparison subtab overlays stacked rectangles across selected criteria and provides a combined table to quantitatively compare ranges.

\vspace{0.5em}
\noindent\textbf{Engineering outcome.} The final ranges provide defensible, transparent envelopes for each DVA parameter, reflecting both central tendency and high-performance concentration while remaining robust to outliers and distributional irregularities.


\section{Design Using Statistical Analysis: A Comprehensive Post-Optimization Framework}
\label{sec:design-statistical-analysis}

\subsection{Motivation and Problem Statement}

Even when all optimization runs achieve successful convergence to the target fitness threshold, the resulting parameter vectors exhibit significant variability across different runs. This variability arises from multiple sources: (i) the stochastic nature of evolutionary algorithms, (ii) the existence of multiple local optima in the high-dimensional DVA parameter space, (iii) the influence of different seeding strategies and adaptive controllers, and (iv) the inherent trade-offs between different performance criteria. Consequently, a single "optimal" solution may not represent the most robust or practical design choice for real-world implementation.

The statistical analysis framework addresses this challenge by providing systematic methods to extract meaningful parameter ranges from multiple optimization runs. These ranges serve as design envelopes that balance performance requirements with manufacturing tolerances, operational robustness, and practical implementation constraints. The framework enables engineers to make informed decisions about parameter selection based on statistical evidence rather than relying on a single point estimate.

\subsection{Design Philosophy and Methodology}

The design process follows a hierarchical approach that integrates statistical rigor with engineering practicality:

\begin{enumerate}
    \item \textbf{Data-Driven Design}: Leverage the statistical distribution of optimized parameters across multiple runs to identify robust design regions
    \item \textbf{Risk-Aware Selection}: Consider both central tendency and variability to balance performance with manufacturing feasibility
    \item \textbf{Multi-Criteria Decision Making}: Integrate different statistical criteria to capture various aspects of parameter behavior
    \item \textbf{Validation and Verification}: Ensure selected ranges maintain performance across different operating conditions
\end{enumerate}

\subsection{Statistical Criteria for Design Decision Making}

\subsubsection{Performance-Based Design Criteria}

\textbf{Elite Performance Concentration (Top-$q$ Analysis):}
For parameters that show strong correlation with fitness performance, the Top-$q$ criterion identifies regions where high-performing solutions concentrate:

\begin{equation}
\mathcal{R}_{\text{elite}}^{(j)} = [L_{\text{Top-$q$}}^{(j)}, U_{\text{Top-$q$}}^{(j)}]
\end{equation}

where:
\begin{align}
L_{\text{Top-$q$}}^{(j)} &= Q_{x_j \in \mathcal{X}^{q}_{p_j}}(0.05) \\
U_{\text{Top-$q$}}^{(j)} &= Q_{x_j \in \mathcal{X}^{q}_{p_j}}(0.95)
\end{align}

This criterion is particularly valuable for:
\begin{itemize}
    \item Critical parameters that directly influence system performance
    \item Design scenarios where performance optimization is the primary objective
    \item Cases where manufacturing precision can accommodate tighter tolerances
\end{itemize}

\textbf{Design Algorithm for Elite Performance Selection:}
\begin{algorithm}[H]
\caption{Elite Performance Parameter Selection}
\begin{algorithmic}[1]
\REQUIRE Parameter-fitness pairs $\{(x_j^{(r)}, f^{(r)})\}_{r=1}^R$ for parameter $p_j$; elite fraction $q$ (default: 0.25)
\STATE Compute fitness threshold $\tau = Q_{\{f^{(r)}\}}(q)$
\STATE Identify elite runs: $\mathcal{I}_{\text{elite}} = \{r : f^{(r)} \leq \tau\}$
\STATE Extract elite parameter values: $\mathcal{X}_{\text{elite}}^{(j)} = \{x_j^{(r)} : r \in \mathcal{I}_{\text{elite}}\}$
\STATE Compute elite range: $[L_{\text{elite}}, U_{\text{elite}}] = [Q_{\mathcal{X}_{\text{elite}}^{(j)}}(0.05), Q_{\mathcal{X}_{\text{elite}}^{(j)}}(0.95)]$
\STATE \textbf{return} Elite range $[L_{\text{elite}}, U_{\text{elite}}]$ and elite sample size $|\mathcal{I}_{\text{elite}}|$
\end{algorithmic}
\end{algorithm}

\subsubsection{Robustness-Based Design Criteria}

\textbf{High-Density Interval Design (HDI):}
The HDI criterion identifies the most compact region containing a specified fraction of solutions, making it ideal for robust design:

\begin{equation}
\mathcal{R}_{\text{HDI}}^{(j)} = [L_{\text{HDI}}^{(j)}, U_{\text{HDI}}^{(j)}]
\end{equation}

where $[L_{\text{HDI}}^{(j)}, U_{\text{HDI}}^{(j)}]$ is the shortest interval containing $\alpha \times 100\%$ of the data.

\textbf{Design Algorithm for HDI Selection:}
\begin{algorithm}[H]
\caption{High-Density Interval Parameter Selection}
\begin{algorithmic}[1]
\REQUIRE Parameter sample $\mathcal{X}^{(j)} = \{x_j^{(r)}\}_{r=1}^R$; coverage fraction $\alpha$ (default: 0.68)
\STATE Sort values: $v_1 \leq v_2 \leq \ldots \leq v_R$
\STATE Compute window size: $k = \lfloor \alpha R \rceil$
\STATE Find shortest window: $(s^*, w^*) = \arg\min_{1 \leq s \leq R-k+1} (v_{s+k-1} - v_s)$
\STATE \textbf{return} HDI range $[v_{s^*}, v_{s^*+k-1}]$ and width $w^*$
\end{algorithmic}
\end{algorithm}

\subsubsection{Manufacturing-Friendly Design Criteria}

\textbf{Interquartile Range Design (IQR):}
The IQR provides a robust, outlier-resistant range suitable for manufacturing:

\begin{equation}
\mathcal{R}_{\text{IQR}}^{(j)} = [Q_{x_j}(0.25), Q_{x_j}(0.75)]
\end{equation}

This criterion is preferred when:
\begin{itemize}
    \item Manufacturing tolerances are moderate
    \item Outlier resistance is important
    \item A balanced compromise between performance and robustness is needed
\end{itemize}

\textbf{Trimmed Mean Design (TMAD):}
For highly variable parameters, the TMAD criterion provides additional robustness:

\begin{equation}
\mathcal{R}_{\text{TMAD}}^{(j)} = [\tilde{\mu} - 1.5\sigma_{\text{rob}}, \tilde{\mu} + 1.5\sigma_{\text{rob}}]
\end{equation}

where:
\begin{align}
\tilde{\mu} &= \text{trimmed mean (central 80\%)} \\
\sigma_{\text{rob}} &= 1.4826 \times \text{MAD}
\end{align}

\subsection{Multi-Criteria Design Decision Framework}

\subsubsection{Consensus-Based Design Selection}

When multiple criteria agree on a parameter range, it indicates high confidence in the design choice. The consensus score quantifies this agreement:

\begin{equation}
S_{\text{cons}}^{(j)} = \frac{\max\{0, U_{\cap}^{(j)} - L_{\cap}^{(j)}\}}{\max\{\epsilon, U_{\cup}^{(j)} - L_{\cup}^{(j)}\}}
\end{equation}

where:
\begin{align}
[L_{\cap}^{(j)}, U_{\cap}^{(j)}] &= \bigcap_{c \in \mathcal{C}} [L_c^{(j)}, U_c^{(j)}] \\
[L_{\cup}^{(j)}, U_{\cup}^{(j)}] &= \bigcup_{c \in \mathcal{C}} [L_c^{(j)}, U_c^{(j)}]
\end{align}

\textbf{Consensus-Based Design Algorithm:}
\begin{algorithm}[H]
\caption{Consensus-Based Parameter Range Selection}
\begin{algorithmic}[1]
\REQUIRE Intervals $[L_c^{(j)}, U_c^{(j)}]$ for all criteria $c \in \mathcal{C}$; consensus threshold $\tau_{\text{cons}}$ (default: 0.3)
\STATE Compute intersection: $[L_{\cap}, U_{\cap}] = [\max_c L_c^{(j)}, \min_c U_c^{(j)}]$
\STATE Compute union: $[L_{\cup}, U_{\cup}] = [\min_c L_c^{(j)}, \max_c U_c^{(j)}]$
\STATE Calculate consensus score: $S_{\text{cons}} = \frac{\max\{0, U_{\cap} - L_{\cap}\}}{\max\{\epsilon, U_{\cup} - L_{\cup}\}}$
\IF{$S_{\text{cons}} \geq \tau_{\text{cons}}$}
    \STATE \textbf{return} Consensus range $[L_{\cap}, U_{\cap}]$ and score $S_{\text{cons}}$
\ELSE
    \STATE \textbf{return} Majority intersection or most robust criterion
\ENDIF
\end{algorithmic}
\end{algorithm}

\subsubsection{Performance-Robustness Trade-off Analysis}

For each parameter, we can construct a performance-robustness trade-off curve:

\begin{equation}
\text{Performance Score} = \frac{1}{|\mathcal{I}_{\text{elite}}|} \sum_{r \in \mathcal{I}_{\text{elite}}} \mathbb{I}[x_j^{(r)} \in \mathcal{R}^{(j)}]
\end{equation}

\begin{equation}
\text{Robustness Score} = 1 - \frac{W^{(j)}}{\max_{c \in \mathcal{C}} W_c^{(j)}}
\end{equation}

where $W^{(j)}$ is the width of the selected range for parameter $p_j$.

\subsection{Design Guidelines and Decision Trees}

\subsubsection{Parameter Classification and Design Strategy}

Based on the statistical analysis, parameters can be classified into different categories requiring distinct design approaches:

\textbf{Category 1: High-Performance Critical Parameters}
\begin{itemize}
    \item \textbf{Characteristics}: Strong correlation with fitness, low variability in elite solutions
    \item \textbf{Design Strategy}: Use Top-$q$ criterion with tight tolerances
    \item \textbf{Manufacturing Requirements}: High precision, strict quality control
    \item \textbf{Validation Approach}: Performance testing under multiple operating conditions
\end{itemize}

\textbf{Category 2: Robust Design Parameters}
\begin{itemize}
    \item \textbf{Characteristics}: Moderate correlation with fitness, high consensus among criteria
    \item \textbf{Design Strategy}: Use consensus-based selection or HDI criterion
    \item \textbf{Manufacturing Requirements}: Standard precision, cost-effective production
    \item \textbf{Validation Approach}: Statistical process control and tolerance analysis
\end{itemize}

\textbf{Category 3: Flexible Design Parameters}
\begin{itemize}
    \item \textbf{Characteristics}: Weak correlation with fitness, high variability across runs
    \item \textbf{Design Strategy}: Use IQR or TMAD criterion with broader tolerances
    \item \textbf{Manufacturing Requirements}: Standard manufacturing processes
    \item \textbf{Validation Approach}: Functional testing and operational verification
\end{itemize}

\subsubsection{Decision Tree for Parameter Range Selection}

\begin{algorithm}[H]
\caption{Parameter Range Selection Decision Tree}
\begin{algorithmic}[1]
\REQUIRE Parameter $p_j$ with sample $\mathcal{X}^{(j)}$ and fitness correlation $\rho_j$
\STATE Compute all statistical criteria intervals
\STATE Calculate consensus score $S_{\text{cons}}^{(j)}$
\IF{$\rho_j > 0.7$ (high performance correlation)}
    \IF{$|\mathcal{I}_{\text{elite}}| \geq 10$ (sufficient elite samples)}
        \STATE Select Top-$q$ range
        \STATE \textbf{return} Elite performance range
    \ELSE
        \STATE Fall back to HDI criterion
        \STATE \textbf{return} HDI range
    \ENDIF
\ELSIF{$S_{\text{cons}}^{(j)} > 0.5$ (high consensus)}
    \STATE Select consensus intersection
    \STATE \textbf{return} Consensus range
\ELSIF{$\text{CV}^{(j)} < 0.3$ (low variability)}
    \STATE Select IQR range
    \STATE \textbf{return} IQR range
\ELSE
    \STATE Select TMAD range for robustness
    \STATE \textbf{return} TMAD range
\ENDIF
\end{algorithmic}
\end{algorithm}

\subsection{Design Validation and Verification}

\subsubsection{Monte Carlo Validation}

To validate the selected parameter ranges, we perform Monte Carlo simulation:

\begin{equation}
\text{Validation Score} = \frac{1}{N_{\text{MC}}} \sum_{i=1}^{N_{\text{MC}}} \mathbb{I}[f(\mathbf{x}^{(i)}) \leq f_{\text{target}}]
\end{equation}

where $\mathbf{x}^{(i)}$ are randomly sampled from the selected parameter ranges.

\textbf{Monte Carlo Validation Algorithm:}
\begin{algorithm}[H]
\caption{Parameter Range Validation via Monte Carlo}
\begin{algorithmic}[1]
\REQUIRE Selected ranges $[L_j, U_j]$ for all parameters; target fitness $f_{\text{target}}$; validation samples $N_{\text{MC}}$
\STATE Initialize success counter: $N_{\text{success}} = 0$
\FOR{$i = 1$ to $N_{\text{MC}}$}
    \STATE Sample parameter vector: $x_j^{(i)} \sim \text{Uniform}(L_j, U_j)$ for all $j$
    \STATE Evaluate fitness: $f^{(i)} = f(\mathbf{x}^{(i)})$
    \IF{$f^{(i)} \leq f_{\text{target}}$}
        \STATE $N_{\text{success}} \leftarrow N_{\text{success}} + 1$
    \ENDIF
\ENDFOR
\STATE Compute validation score: $V = \frac{N_{\text{success}}}{N_{\text{MC}}}$
\STATE \textbf{return} Validation score $V$ and confidence interval
\end{algorithmic}
\end{algorithm}

\subsubsection{Sensitivity Analysis for Design Robustness}

We assess the sensitivity of the design to parameter variations:

\begin{equation}
S_j = \frac{\partial f}{\partial x_j} \approx \frac{f(\mathbf{x} + \Delta x_j \mathbf{e}_j) - f(\mathbf{x})}{\Delta x_j}
\end{equation}

where $\mathbf{e}_j$ is the unit vector in the $j$-th direction.

\subsection{Engineering Implementation Guidelines}

\subsubsection{Manufacturing Tolerance Specification}

Based on the statistical ranges, manufacturing tolerances can be specified as:

\begin{equation}
\text{Tolerance}_j = \frac{U_j - L_j}{2} \times \text{Safety Factor}
\end{equation}

where the safety factor accounts for manufacturing variability and operational uncertainties.

\subsubsection{Quality Control Procedures}

\textbf{Statistical Process Control (SPC):}
\begin{itemize}
    \item Monitor parameter values during manufacturing using control charts
    \item Set control limits based on the selected statistical ranges
    \item Implement corrective actions when parameters drift outside acceptable ranges
\end{itemize}

\textbf{Acceptance Testing:}
\begin{itemize}
    \item Test manufactured DVAs using the same fitness evaluation as optimization
    \item Accept components within the specified parameter ranges
    \item Document performance characteristics for continuous improvement
\end{itemize}

\subsection{Advanced Design Techniques}

\subsubsection{Multi-Objective Design Optimization}

For complex design scenarios, we can extend the framework to handle multiple objectives:

\begin{equation}
\text{Design Score} = w_1 \times \text{Performance Score} + w_2 \times \text{Robustness Score} + w_3 \times \text{Cost Score}
\end{equation}

where weights $w_i$ reflect the relative importance of each objective.

\subsubsection{Adaptive Design Refinement}

The design process can be iterative, using feedback from manufacturing and testing:

\begin{algorithm}[H]
\caption{Adaptive Design Refinement}
\begin{algorithmic}[1]
\REQUIRE Initial parameter ranges; manufacturing feedback; performance data
\STATE Collect manufacturing data and performance measurements
\STATE Analyze correlation between parameter variations and performance
\STATE Adjust parameter ranges based on empirical evidence
\STATE Re-run validation and update design specifications
\STATE \textbf{return} Refined parameter ranges
\end{algorithmic}
\end{algorithm}

\subsection{Case Study: Complete DVA Design Workflow}

\subsubsection{Step-by-Step Design Process}

\textbf{Step 1: Optimization Execution}
\begin{enumerate}
    \item Execute multiple optimization runs with different seeds and strategies
    \item Collect parameter vectors and fitness values from all runs
    \item Validate convergence and solution quality
\end{enumerate}

\textbf{Step 2: Statistical Analysis}
\begin{enumerate}
    \item Compute all statistical criteria for each parameter
    \item Analyze parameter-fitness correlations
    \item Identify parameter categories and design strategies
\end{enumerate}

\textbf{Step 3: Range Selection}
\begin{enumerate}
    \item Apply decision tree for each parameter
    \item Consider manufacturing constraints and cost implications
    \item Select final parameter ranges
\end{enumerate}

\textbf{Step 4: Validation and Verification}
\begin{enumerate}
    \item Perform Monte Carlo validation
    \item Conduct sensitivity analysis
    \item Verify performance under different operating conditions
\end{enumerate}

\textbf{Step 5: Manufacturing Specification}
\begin{enumerate}
    \item Define manufacturing tolerances
    \item Establish quality control procedures
    \item Document design specifications
\end{enumerate}

\subsubsection{Design Documentation and Traceability}

The design process should be fully documented to ensure traceability and reproducibility:

\begin{itemize}
    \item \textbf{Optimization Records}: Complete configuration and results from all runs
    \item \textbf{Statistical Analysis}: Detailed computation of all criteria and metrics
    \item \textbf{Decision Rationale}: Justification for selected parameter ranges
    \item \textbf{Validation Results}: Monte Carlo and sensitivity analysis outcomes
    \item \textbf{Manufacturing Specifications}: Tolerances, quality control procedures
\end{itemize}

\subsection{Software Implementation and User Interface}

The statistical analysis framework is implemented in the \softwareName{} GUI with the following features:

\subsubsection{Parameter Ranges Tab}
\begin{itemize}
    \item Individual tabs for each statistical criterion (IQR, P5--P95, Tukey, HDI, Top-$q$, TMAD)
    \item Interactive tables showing ranges, widths, and centers for all parameters
    \item Export functionality for design documentation
\end{itemize}

\subsubsection{Compare Criteria Tab}
\begin{itemize}
    \item Side-by-side comparison of different criteria
    \item Consensus analysis and agreement metrics
    \item Visual overlays showing interval relationships
\end{itemize}

\subsubsection{Design Recommendations Tab}
\begin{itemize}
    \item Automated parameter classification and strategy selection
    \item Decision tree implementation with user override options
    \item Manufacturing tolerance calculations
    \item Validation score computation
\end{itemize}

\section{DeVana: The First Comprehensive DVA Design Playground}
\label{sec:devana-software}

\subsection{Introduction to DeVana: A Revolutionary Design Platform}

\softwareName{} (\softwareVersion{}) represents the first-of-its-kind comprehensive software platform specifically designed for Dynamic Vibration Absorber (DVA) optimization and statistical-based design. Unlike existing optimization tools that focus on generic problems or require extensive customization, \softwareName{} provides a specialized, integrated environment that unifies mechanical modeling, advanced optimization algorithms, statistical analysis, and design synthesis under a single, user-friendly interface.

The platform's revolutionary approach addresses a critical gap in the vibration control engineering community: the lack of a dedicated, reproducible, and extensible framework for DVA design that combines classical optimization methods with modern statistical analysis techniques. \softwareName{} serves as both a research tool for advancing the state-of-the-art in vibration control and a practical engineering platform for real-world DVA design projects.

\subsection{Core Design Philosophy and Architecture}

\subsubsection{Unified Optimization Playground}

\softwareName{} implements a unique "optimization playground" concept where designers can experiment with different optimization strategies on identical mechanical problems. This playground approach enables:

\begin{itemize}
    \item \textbf{Algorithm Comparison}: Direct comparison of \gls{ga}, \gls{pso}, \gls{sa}, \gls{cmaes}, \gls{rl}, and \gls{de} on the same DVA configuration
    \item \textbf{Reproducible Research}: Versioned configurations and deterministic seeds ensure reproducible results across different research groups
    \item \textbf{Extensible Framework}: Clean interfaces allow easy addition of new optimizers, surrogates, and performance metrics
    \item \textbf{Statistical Synthesis}: Built-in statistical analysis tools for extracting robust parameter ranges from multiple optimization runs
\end{itemize}

\subsubsection{Modular Software Architecture}

The software follows a modular, layered architecture that separates concerns and promotes maintainability:

\begin{enumerate}
    \item \textbf{Mechanical Core Layer}: Implements the 2DOF-3DOF system dynamics, FRF computation, and performance evaluation
    \item \textbf{Optimization Layer}: Houses the algorithm suite with consistent interfaces and instrumentation
    \item \textbf{Controller Layer}: Manages adaptive controllers (ML-bandit, RL) for online parameter tuning
    \item \textbf{Surrogate Layer}: Provides KNN-based screening and NeuralSeeder capabilities
    \item \textbf{Statistical Analysis Layer}: Implements the comprehensive range synthesis framework
    \item \textbf{GUI Layer}: Offers intuitive user interface with real-time visualization and control
\end{enumerate}

\subsection{Comprehensive Algorithm Suite}

\subsubsection{Traditional Optimization Methods}

\softwareName{} includes well-established optimization algorithms, each carefully adapted for DVA design:

\textbf{Genetic Algorithm (GA):}
\begin{itemize}
    \item Real-valued representation tailored to DVA parameter vectors
    \item Blend crossover (BLX-α) with adaptive mutation rates
    \item Tournament selection with configurable pressure
    \item Generational replacement with elitism
\end{itemize}

\textbf{Particle Swarm Optimization (PSO):}
\begin{itemize}
    \item Velocity and position updates with inertia weight adaptation
    \item Global and local best memory mechanisms
    \item Boundary handling through reflection and clamping
    \item Configurable swarm topology and communication
\end{itemize}

\textbf{Simulated Annealing (SA):}
\begin{itemize}
    \item Metropolis acceptance criterion with temperature scheduling
    \item Gaussian perturbation with adaptive step sizes
    \item Restart mechanisms for escaping local optima
    \item Multi-start capability for global exploration
\end{itemize}

\textbf{Differential Evolution (DE):}
\begin{itemize}
    \item Multiple mutation strategies (DE/rand/1, DE/best/1, DE/current-to-best/1)
    \item Adaptive crossover and mutation rates
    \item Population diversity maintenance
    \item Constraint handling through projection
\end{itemize}

\subsubsection{Advanced Optimization Methods}

\textbf{Covariance Matrix Adaptation Evolution Strategy (CMA-ES):}
\begin{itemize}
    \item Self-adaptive step size and covariance matrix updates
    \item Weighted recombination of selected individuals
    \item Rank-based selection with elitism
    \item Automatic parameter adaptation for different problem scales
\end{itemize}

\textbf{Reinforcement Learning (RL) Controller:}
\begin{itemize}
    \item Q-learning agent with ε-greedy exploration
    \item State space encoding of optimization progress
    \item Action space for crossover, mutation, and population adaptation
    \item Reward shaping for balanced exploration-exploitation
\end{itemize}

\subsection{Adaptive Control and Learning Mechanisms}

\subsubsection{Machine Learning Bandit Controller}

The ML-bandit controller represents a novel approach to online optimization parameter adaptation:

\begin{equation}
a_t = \arg\max_{a \in \mathcal{A}} \left[ \hat{R}_t(a) + c \sqrt{\frac{\ln t}{N_t(a)}} \right]
\end{equation}

where:
\begin{itemize}
    \item $\hat{R}_t(a)$: Empirical average reward for action $a$ at time $t$
    \item $N_t(a)$: Number of times action $a$ has been selected
    \item $c$: Exploration constant balancing exploitation vs. exploration
\end{itemize}

\textbf{Action Space Definition:}
\begin{align}
\mathcal{A} &= \{(\delta_{\text{cx}}, \delta_{\mu}, \pi) : \delta_{\text{cx}}, \delta_{\mu} \in \{-0.2, -0.1, 0, 0.1, 0.2\}, \\
&\quad \pi \in \{0.8, 0.9, 1.0, 1.1, 1.2\}\}
\end{align}

\textbf{Reward Function:}
\begin{equation}
R_t = \frac{\max\{0, f_{t-1}^* - f_t^*\}}{\max\{\epsilon, T_t\}} \cdot \frac{1}{\max\{1, E_t\}} - w_{\text{div}} |\text{CV}_t - \text{CV}^{\text{target}}|
\end{equation}

\subsubsection{Reinforcement Learning Controller}

The RL controller implements a compact Q-learning agent:

\begin{equation}
Q(s, a) \leftarrow Q(s, a) + \alpha [R + \gamma \max_{a'} Q(s', a') - Q(s, a)]
\end{equation}

where:
\begin{itemize}
    \item $\alpha$: Learning rate (default: 0.1)
    \item $\gamma$: Discount factor (default: 0.9)
    \item $\epsilon$: Exploration rate with exponential decay
\end{itemize}

\subsection{Advanced Seeding and Surrogate Methods}

\subsubsection{Quasi-Monte Carlo Seeding}

\softwareName{} implements sophisticated seeding strategies for improved initial population quality:

\textbf{Sobol Sequence:}
\begin{itemize}
    \item Low-discrepancy sequence for uniform space coverage
    \item Deterministic generation ensuring reproducibility
    \item Superior uniformity compared to random sampling
\end{itemize}

\textbf{Latin Hypercube Sampling (LHS):}
\begin{itemize}
    \item Stratified sampling ensuring each dimension is uniformly covered
    \item Reduced variance in initial population distribution
    \item Configurable sample size and dimension handling
\end{itemize}

\subsubsection{Neural Network-Based Seeding (NeuralSeeder)}

The NeuralSeeder represents a breakthrough in intelligent population initialization:

\begin{equation}
\text{UCB}(\mathbf{x}) = \mu(\mathbf{x}) - \beta \sigma(\mathbf{x})
\end{equation}

\begin{equation}
\text{EI}(\mathbf{x}) = \mathbb{E}[\max\{0, f^* - Y(\mathbf{x})\}]
\end{equation}

where:
\begin{itemize}
    \item $\mu(\mathbf{x})$: Predicted mean from neural ensemble
    \item $\sigma(\mathbf{x})$: Predicted uncertainty
    \item $\beta$: Exploration-exploitation balance parameter
    \item $f^*$: Current best fitness value
\end{itemize}

\textbf{Neural Ensemble Architecture:}
\begin{itemize}
    \item Multi-layer perceptron with configurable hidden layers
    \item Dropout regularization for uncertainty quantification
    \item Ensemble of multiple models for robust predictions
    \item Gradient-based refinement for high-quality proposals
\end{itemize}

\subsubsection{Surrogate-Assisted Screening}

The KNN-based surrogate reduces computational cost while maintaining solution quality:

\begin{equation}
\hat{f}(\tilde{\mathbf{x}}) = \frac{1}{k} \sum_{(\mathbf{x}^{(n)}, y^{(n)}) \in \mathcal{N}_k(\tilde{\mathbf{x}})} y^{(n)}
\end{equation}

where $\mathcal{N}_k(\tilde{\mathbf{x}})$ represents the $k$ nearest neighbors in normalized parameter space.

\textbf{Novelty-Based Exploration:}
\begin{equation}
\mathcal{N}(\tilde{\mathbf{x}}) = \min_{\mathbf{x}^{(n)} \in \mathcal{D}} \|\tilde{\mathbf{z}} - \mathbf{z}^{(n)}\|_2
\end{equation}

\subsection{Statistical Analysis and Design Synthesis}

\subsubsection{Comprehensive Range Synthesis Framework}

\softwareName{} implements the complete statistical analysis framework described in Section~\ref{sec:statistical-ranges}, providing:

\textbf{Six Statistical Criteria:}
\begin{enumerate}
    \item \textbf{IQR}: Interquartile range for robust, outlier-resistant ranges
    \item \textbf{P5--P95}: 5th to 95th percentile for broad coverage
    \item \textbf{Tukey}: Classical outlier trimming using 1.5×IQR rule
    \item \textbf{HDI}: Highest density interval for skewed distributions
    \item \textbf{Top-$q$}: Elite performance concentration analysis
    \item \textbf{TMAD}: Trimmed mean with median absolute deviation
\end{enumerate}

\textbf{Consensus Analysis:}
\begin{equation}
S_{\text{cons}}^{(j)} = \frac{\max\{0, U_{\cap}^{(j)} - L_{\cap}^{(j)}\}}{\max\{\epsilon, U_{\cup}^{(j)} - L_{\cup}^{(j)}\}}
\end{equation}

\subsubsection{Design Decision Support}

The software provides automated design decision support through:

\textbf{Parameter Classification:}
\begin{itemize}
    \item Automatic correlation analysis between parameters and fitness
    \item Classification into performance-critical, robust, and flexible parameters
    \item Recommended design strategies for each parameter category
\end{itemize}

\textbf{Manufacturing Tolerance Calculation:}
\begin{equation}
\text{Tolerance}_j = \frac{U_j - L_j}{2} \times \text{Safety Factor}
\end{equation}

\subsection{User Interface and Design Workflow}

\subsubsection{Intuitive GUI Design}

The \softwareName{} GUI provides an intuitive, workflow-oriented interface:

\textbf{Main Configuration Panel:}
\begin{itemize}
    \item System parameter configuration with bounds and fixed parameter handling
    \item Optimization algorithm selection with parameter tuning
    \item Objective function weighting and target specification
    \item Seeding strategy configuration
\end{itemize}

\textbf{Real-Time Monitoring:}
\begin{itemize}
    \item Live convergence plots with multiple metrics
    \item Population diversity visualization
    \item Resource usage monitoring (CPU, memory, I/O)
    \item Controller action history and adaptation tracking
\end{itemize}

\textbf{Results Analysis:}
\begin{itemize}
    \item Interactive parameter range visualization
    \item Statistical criteria comparison tools
    \item Export functionality for design documentation
    \item Benchmark results aggregation and comparison
\end{itemize}

\subsubsection{Design Workflow Integration}

\softwareName{} supports the complete DVA design workflow:

\textbf{Step 1: System Configuration}
\begin{enumerate}
    \item Define mechanical system parameters and bounds
    \item Specify target performance criteria and weights
    \item Configure optimization algorithm and parameters
    \item Select seeding and surrogate strategies
\end{enumerate}

\textbf{Step 2: Optimization Execution}
\begin{enumerate}
    \item Execute multiple runs with different configurations
    \item Monitor convergence and performance in real-time
    \item Collect and aggregate results from all runs
    \item Validate solution quality and convergence
\end{enumerate}

\textbf{Step 3: Statistical Analysis}
\begin{enumerate}
    \item Compute all statistical criteria automatically
    \item Analyze parameter-fitness correlations
    \item Generate parameter classification and recommendations
    \item Perform consensus analysis across criteria
\end{enumerate}

\textbf{Step 4: Design Synthesis}
\begin{enumerate}
    \item Select final parameter ranges using decision tree
    \item Calculate manufacturing tolerances
    \item Generate design documentation and specifications
    \item Export results for implementation
\end{enumerate}

\subsection{Codebase Architecture and Implementation}

\subsubsection{Core Implementation Details}

\softwareName{} is implemented in Python 3.8+ with the following key components:

\textbf{Mechanical Modeling (\texttt{codes/Continues\_beam/}):}
\begin{itemize}
    \item \texttt{beam/fem.py}: Finite element implementation of 2DOF-3DOF system
    \item \texttt{beam/solver.py}: Frequency response function computation
    \item \texttt{beam/properties.py}: Material and geometric property handling
\end{itemize}

\textbf{Optimization Algorithms (\texttt{codes/workers/}):}
\begin{itemize}
    \item \texttt{GAWorker.py}: Comprehensive genetic algorithm implementation
    \item \texttt{PSOWorker.py}: Particle swarm optimization
    \item \texttt{SAWorker.py}: Simulated annealing
    \item \texttt{CMAESWorker.py}: CMA-ES implementation
    \item \texttt{DEWorker.py}: Differential evolution
\end{itemize}

\textbf{GUI Framework (\texttt{codes/gui/}):}
\begin{itemize}
    \item \texttt{main\_window/}: Main application interface
    \item \texttt{ga\_mixin.py}: Genetic algorithm specific functionality
    \item \texttt{plotwindow.py}: Visualization and plotting capabilities
\end{itemize}

\subsubsection{Key Software Features}

\textbf{Modularity and Extensibility:}
\begin{itemize}
    \item Clean separation of concerns between layers
    \item Plugin architecture for new optimizers and surrogates
    \item Consistent interfaces across all components
    \item Comprehensive error handling and logging
\end{itemize}

\textbf{Performance and Scalability:}
\begin{itemize}
    \item Multi-threaded optimization execution
    \item Efficient numerical computations using NumPy/SciPy
    \item Memory-efficient data structures for large-scale problems
    \item Parallel processing support for multiple runs
\end{itemize}

\textbf{Reproducibility and Documentation:}
\begin{itemize}
    \item Versioned configurations with hash-based identification
    \item Deterministic random number generation
    \item Comprehensive logging and metrics collection
    \item Export capabilities for all results and configurations
\end{itemize}

\subsection{Comparison with Existing Tools}

\subsubsection{Unique Features of DeVana}

\softwareName{} distinguishes itself from existing optimization and engineering software through several unique features:

\textbf{Specialized DVA Design:}
\begin{itemize}
    \item First software specifically designed for DVA optimization
    \item Built-in mechanical models for 2DOF-3DOF systems
    \item Specialized fitness functions for vibration control objectives
    \item Domain-specific parameter handling and constraints
\end{itemize}

\textbf{Statistical Design Framework:}
\begin{itemize}
    \item Integrated statistical analysis for parameter range synthesis
    \item Multiple criteria for robust design decision making
    \item Automated design recommendations and tolerance calculation
    \item Consensus analysis for confidence assessment
\end{itemize}

\textbf{Advanced Optimization Playground:}
\begin{itemize}
    \item Unified interface for multiple optimization algorithms
    \item Adaptive controllers for online parameter tuning
    \item Surrogate-assisted methods for computational efficiency
    \item Comprehensive benchmarking and comparison tools
\end{itemize}

\subsubsection{Advantages Over Generic Optimization Tools}

Compared to generic optimization software (e.g., MATLAB Optimization Toolbox, Python scipy.optimize):

\begin{itemize}
    \item \textbf{Domain Expertise}: Built-in knowledge of DVA design requirements
    \item \textbf{Specialized Algorithms}: Optimized implementations for vibration control problems
    \item \textbf{Integrated Workflow}: Complete design process from optimization to manufacturing
    \item \textbf{Statistical Analysis}: Built-in tools for robust design synthesis
    \item \textbf{Reproducibility}: Versioned configurations and deterministic execution
\end{itemize}

\subsection{Research and Educational Applications}

\subsubsection{Research Contributions}

\softwareName{} serves as a platform for advancing research in:

\textbf{Vibration Control:}
\begin{itemize}
    \item Novel DVA configurations and optimization strategies
    \item Adaptive control algorithms for dynamic systems
    \item Multi-objective optimization in mechanical design
    \item Robust design methodologies for uncertain systems
\end{itemize}

\textbf{Optimization Algorithms:}
\begin{itemize}
    \item Adaptive parameter control in evolutionary algorithms
    \item Surrogate-assisted optimization for expensive functions
    \item Reinforcement learning in optimization contexts
    \item Statistical analysis of optimization results
\end{itemize}

\subsubsection{Educational Value}

The platform provides educational benefits through:

\textbf{Hands-on Learning:}
\begin{itemize}
    \item Interactive exploration of optimization algorithms
    \item Real-time visualization of convergence behavior
    \item Comparison of different optimization strategies
    \item Understanding of statistical design principles
\end{itemize}

\textbf{Research Training:}
\begin{itemize}
    \item Reproducible experimental setup
    \item Benchmarking and comparison methodologies
    \item Statistical analysis and interpretation
    \item Documentation and reporting practices
\end{itemize}

\subsection{Future Development and Extensibility}

\subsubsection{Planned Enhancements}

The \softwareName{} development roadmap includes:

\textbf{Algorithm Extensions:}
\begin{itemize}
    \item Additional optimization algorithms (NSGA-II, MOEA/D)
    \item Advanced surrogate models (Gaussian processes, neural networks)
    \item Multi-objective optimization capabilities
    \item Constraint handling methods
\end{itemize}

\textbf{Mechanical Modeling:}
\begin{itemize}
    \item Support for more complex DVA configurations
    \item Nonlinear system modeling capabilities
    \item Multi-physics coupling (structural, thermal, fluid)
    \item Real-time system identification
\end{itemize}

\textbf{User Interface:}
\begin{itemize}
    \item Web-based interface for remote access
    \item Mobile application for monitoring and control
    \item Advanced visualization and virtual reality integration
    \item Collaborative design features
\end{itemize}

\subsubsection{Community and Open Source}

\softwareName{} is developed as an open-source project to:

\begin{itemize}
    \item Foster collaboration in the vibration control community
    \item Enable reproducible research and fair comparisons
    \item Provide educational resources for students and researchers
    \item Accelerate innovation in DVA design methodologies
\end{itemize}

\subsection{Conclusion and Impact}

\softwareName{} represents a paradigm shift in DVA design, providing the first comprehensive platform that unifies optimization, statistical analysis, and design synthesis. The software's unique combination of specialized algorithms, adaptive controllers, and statistical design framework enables engineers and researchers to:

\begin{enumerate}
    \item \textbf{Explore Design Space}: Systematically investigate DVA configurations using multiple optimization strategies
    \item \textbf{Make Informed Decisions}: Use statistical evidence to select robust parameter ranges
    \item \textbf{Ensure Reproducibility}: Maintain versioned configurations and deterministic execution
    \item \textbf{Accelerate Innovation}: Focus on design rather than algorithm implementation
    \item \textbf{Enable Collaboration}: Share configurations and results across research groups
\end{enumerate}

The platform's impact extends beyond individual DVA design projects to advance the entire field of vibration control engineering. By providing a standardized, reproducible framework for DVA optimization and design, \softwareName{} enables fair comparisons between different approaches, accelerates research progress, and facilitates the transfer of academic innovations to industrial applications.

As the first software of its kind, \softwareName{} establishes a new standard for DVA design methodology and serves as a foundation for future developments in vibration control optimization. The platform's open-source nature ensures that the benefits of this research are widely accessible to the engineering community, promoting collaboration and accelerating the advancement of vibration control technology.

\subsection{Conclusion and Best Practices}

The statistical analysis framework provides a systematic approach to DVA design that balances performance optimization with practical implementation constraints. Key best practices include:

\begin{enumerate}
    \item \textbf{Multiple Optimization Runs}: Always perform multiple runs to capture solution variability
    \item \textbf{Statistical Rigor}: Use appropriate criteria based on parameter characteristics
    \item \textbf{Engineering Judgment}: Combine statistical analysis with domain expertise
    \item \textbf{Validation and Verification}: Always validate selected ranges before implementation
    \item \textbf{Documentation}: Maintain complete records for traceability and continuous improvement
\end{enumerate}

This framework enables engineers to make informed design decisions based on statistical evidence while considering practical manufacturing and operational constraints. The result is a robust, manufacturable DVA design that achieves the desired performance objectives with acceptable variability and cost.



\end{document} 